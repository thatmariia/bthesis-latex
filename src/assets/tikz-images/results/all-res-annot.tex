% ----- INPUT
\newcommand{\resphw}{\textwidth}
% 0.7829 * width
\newcommand{\resphh}{0.7829\textwidth}

% to shift for ticks
% \newcommand{\resshiftx}{1.3em}
% \newcommand{\resshifty}{1em}
\newcommand{\resshiftx}{0.65em}
\newcommand{\resshifty}{0.7em}

% to shift to the middle from kinda the middle
\newcommand{\resmidshiftx}{-0.5em}
\newcommand{\resmidshifty}{0.4em}

%
\newcommand{\reshormid}{0.5 * \resphw}
\newcommand{\resvermid}{0.5 * \resphh}


\newcommand{\resplot}[1]{
\begin{tikzpicture}[
        arr/.style = { -{Stealth[ ]} },
        bluearrow/.style = {arr, draw=third-color, fill=third-color, thick},
        blackarrow/.style = {arr, ultra thick},
    ]
    
    \begin{scope}

        \node[
            shift={(\resmidshiftx, -\resshifty)}
        ] at (\reshormid, 0) {{ \footnotesize Contrast}};
        
        \node[
            shift={(-\resshiftx, \resmidshifty)},
            rotate=90
        ] at (0, \resvermid) {{\footnotesize Coarseness}};

        
        % \node[shift={(\apwest, \apshifth)}, anchor=west] at (0, \appeakh) {peak};
        
    \end{scope}
    
    \begin{scope}
        \node[anchor=south west,inner sep=0] at (0,0) {\includegraphics[height=\resphh]{#1}};
    \end{scope}
        
\end{tikzpicture}
}
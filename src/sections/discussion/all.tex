\section{Discussion}

The present thesis involved simulations of an oscillatory network of PING oscillators to investigate figure-ground segregation, which is the visual process of separating objects from their surroundings. The simulations used texture stimuli representing non-overlapping congruent, evenly spaced Gabor wavelets (annuli) arranged in rectangular grids. The figures in the simulations were defined by having a more uniform contrast in the wavelets than the background. We have manipulated the contrast and coarseness of the figures' annuli to investigate their impact on the synchronization of gamma rhythms in the oscillatory network. Finally, we compared the results of the simulations to experimental data.

From Figures \ref{fig:three-rasters} and \ref{fig:sim-res}, it is clear that the synchronization levels of the oscillatory network are typically higher when the stimuli have a smaller distance between the annuli and a lower contrast of the wavelets. This is consistent with the TWCO, which suggests that the network synchronizes more easily when there is a smaller detuning and a higher coupling strength, as the contrast heterogeneity and grid coarseness correspond to these parameters. Therefore, the results indicate the presence of the Arnold tongue, which can also be visually observed.
As the behavioral experiments mimic the Arnold tongue as well, it is conceivable that the synchronization of the oscillatory network of PING networks could underlie figure-ground segregation.

To test the above statement, we conducted various correlation tests and regression analyses. Although the data only consisted of 25 data points, it already showed a strong monotonic relationship. It is possible that this relationship is linear, as the linearity test revealed a higher statistic. However, it was not possible to draw a significant conclusion about the linearity.
The regression analyses showed that there are both linear and quadratic polynomials that fit the data moderately well. 

We then used bivariate splines to extrapolate the data and increase its size while preserving its intrinsic patterns and variabilities. This revealed an even stronger relationship, with a higher monotonicity statistic than linearity. This was confirmed by the regression analyses, which showed that quadratic and cubic curves fit the data similarly better than a linear function. So, even with the intrinsic variations of the data preserved, the correlation between the experimental and simulations data is strong, although likely not linear.

We also interpolated the data with a 2D psychometric sigmoid function, which removed the intrinsic variabilities and created smooth linearized transitions in the data. The correlation between the resulting datasets was shown to be very strong. The regression analyses showed a very good fit for linear, quadratic, and cubic polynomials with minimal errors. So, when the intrinsic variabilities are removed from the data, the correlation between the experimental and simulation data is very strong.
If one is to predict human performance in figure-ground segregation by analyzing the synchronization of the oscillatory network proposed in this thesis, using a fitted cubic polynomial obtained with the psychometric interpolation would produce the most accurate results among other inspected curves.

While the results of the correlation between the datasets interpolated with a psychometric function suggest a very strong relationship, they do not by themselves provide reliable proof that the model investigated in this thesis can predict human behavior. The probability of generating a matrix of random numbers that, when interpolated with a sigmoid, would show a pattern resembling an Arnold tongue is non-zero.
However, a random generator is clearly not a predictor of figure-ground segregation performance in humans. Therefore, we can only draw conclusions from this result in conjunction with the other two results. The fact that we ran the simulations five times and obtained low variability in the individual results suggests that the averaged results we performed the tests on likely did not occur due to chance.
Since the statistical methods revealed a statistically significant correlation between the original and spline-interpolated data in addition to the data interpolated with the psychometric functions, we can conclude that it is likely that the synchronization of the gamma rhythms in the oscillatory network of PING networks in V1 is indicative of figure-ground segregation.

It is essential to point out that there is a difference in the shapes of Arnold tongues for the simulations and experimental data, as can be best observed from Figures \ref{fig:sim-res-sigmoid} and \ref{fig:exp-res-sigmoid}. Humans appear to be more sensitive to the increase of contrast in a figure than grid coarseness, while it is not strongly evident from the predictions made by the simulated oscillatory network. Although outside of the scope of this thesis, we performed simulations with higher coupling weights between excitatory neurons, and they produced a result that resembled behavioral data more closely. These results are not included in the thesis as the biological plausibility of the used parameters has not been studied.

Synchronization of oscillatory PING networks is influenced by the variability of the current injected into neurons belonging to coupled PING networks, and the magnitude of the current depends on factors such as coupling weights between different types of neurons, synaptic transmission, and noise. The values for the parameters used in this thesis were collected from multiple studies involving monkeys as well as humans. Therefore, there is a potential for finding biologically plausible parameters that would yield results even more indicative of human behavior.

Additionally, brain circuits are not limited to small patches of oscillatory networks in V1. However, the present study examines oscillatory networks in isolation without considering interactions with other neurons in the brain. It may be possible to include some of the effects of interaction with other oscillators in the brain by adjusting pairwise distances between PING networks to account for boundary conditions. Considering interactions with neurons outside of the neurons in the PING networks of the stimuli's receptive field could produce results that match the results of behavioral experiments more closely.

In summary, this thesis has explored the potential of using the synchronization of gamma rhythms in oscillatory networks of PING networks in V1 to predict human performance in figure-ground segregation tasks. We have provided compelling evidence that the proposed model can be a valuable predictor of figure-ground segregation, providing a new approach for predicting human behavior in such tasks. In particular, through model simulations, we have demonstrated a relationship between the synchronization levels of the oscillatory network and the contrast and coarseness in texture stimuli. This relationship is supported by the presence of the Arnold tongue. Through comparison of the simulation results with experimental data, which resembles an Arnold tongue as well, we discovered a strong correlation between the results from the model proposed in this thesis and behavioral data. Additional research is advised to enhance the model's capability to predict human performance in figure-ground segregation more reliably.
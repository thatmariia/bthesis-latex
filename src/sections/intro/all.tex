\section{Introduction}
\label{sec:introduction}

\stimfigcap\footnote{
    Because of the polysemous nature of the word \say{figure}, it is written in italics when referring to the \stimfig{} within a stimulus.
}-ground segregation is a concept that refers to the way the brain organizes visual information into distinct objects and backgrounds. This process allows us to perceive and understand the world by organizing complex visual stimuli into meaningful objects and backgrounds. In this process, the \stimfig{} is the object or part of the visual stimulus that is perceived as the main subject, while the ground is the surrounding context or background. Researchers in the field of perception have long been interested in understanding how the brain processes and organizes visual information through \stimfig-ground segregation \cite{Julesz1983, Julesz1984, Nothdurft1985a, Williams1992, Wagemans2012}.
Many studies have focused on the mechanisms underlying this process using visual texture stimuli. In particular, they investigated the role of a single feature, such as contrast, spatial frequency, color, orientation, or movement direction, in determining the difference between the \stimfig{} and background \cite{Hadjipapas2015, Bredfeldt2002, Henriksson2008, Shapley2011, Lamme1995}. Such an approach allows for simpler segregation than analyzing textures defined by multiple features \cite{Sagi1984, Treisman1980}.

The degree to which texture elements differ in a \stimfig{} and background affects the speed and accuracy of \stimfig-ground segregation \cite{Nothdurft1985a, Nothdurft1985b, Landy1991, Motoyoshi1999, Nothdurft1991a, Nothdurft1991b, DeWeerd1992}.
In particular, studies have shown that the extent of difference between elements on a specific dimension, as well as their relative proximity within the \stimfig, play a role in this process \cite{MaryamPLACEHOLDER}. Altering the \stimfig{} in these manners provides essential information to the neural mechanisms responsible for combining elements that make up that \stimfig{} and segregate it from the background. It has been proposed that these mechanisms are governed by the ability of the neuronal populations' oscillations in low-level visual areas to synchronize in the gamma frequency range \cite{Malsburg1995, Singer1995a, Gray1999}.

Brain oscillations are repetitive patterns of neural activity that can be observed in the brain. These patterns occur at different frequencies and have been linked to a range of mental states and cognitive processes \cite{BuzsakiBook2006:5}. Notably, gamma brain waves (25-80 Hz) are believed to be involved in perception \cite{Eckhorn1988, Gray1989, Kreiter1996, Livingstone1996, Fries1997, Gail2000}. The mechanisms that give rise to gamma oscillations are not well understood, but it is thought that they result from the synchronized activity of neurons in the brain \cite{Wang1996, Buzsaki2012, Fries2015}. This synchronization may be influenced by the interaction of inhibitory and excitatory neurons, as well as the action of neurotransmitters \cite{Fries2007, Tiesinga2009, Tiesinga2010}. 

Synchronization in networks of weakly coupled oscillators (WCOs) is influenced by two factors: the degree to which the intrinsic frequencies of the oscillators differ (detuning) and the strength of their coupling \cite{Pikovsky2002, Tiesinga2010, Lowet2015}. If the coupling strength is strong enough, the oscillators will synchronize, with the critical value for synchronization depending on the detuning. The region of synchronization in the parameter space is referred to as \say{Arnold tongue} due to its triangular shape.
Neural computational models often depict local neuronal populations as oscillators, as they can produce gamma rhythms in response to local stimulation \cite{IzhikevichBook2004:9}. The strength of synchronization between them depends on the two characteristics mentioned above. The contrast between elements in a texture stimulus impacts the frequency detuning \cite{Lowet2017, Ray2010}, and the proximity of those elements affects the coupling strength \cite{Gilbert1983, Tso1986a, Lowet2015, Lowet2017, Stettler2002, MaryamPLACEHOLDER}. Hence, these parameters influence local synchronization between neurons, as confirmed by previous studies \cite{Lowet2017, MaryamPLACEHOLDER}. Section \ref{sec:twco} gives a more in-depth description of the effects of stimuli features on neural synchronization.

A recent study by Karimian has proposed a model of the visual cortex using Kuramoto oscillators and Gabor texture stimuli with varying contrast heterogeneity among texture elements and their coarseness \cite{MaryamPLACEHOLDER}. In these textures, a \stimfig{} region was defined by less heterogeneous contrasts compared to the background.
They then used this model to evaluate the extent to which synchrony is predictive of human performance in a \stimfig-ground segregation experiment. The results showed that human \stimfig-ground segregation performance with the same stimuli conformed to model simulations. This suggests that synchrony of gamma oscillations is predictive of human performance in \stimfig-ground segregation. 

The Pyramidal Interneuron Network Gamma (PING) model is a computational model that seeks to explain the origins of gamma oscillations in the brain. It proposes that these oscillations are generated by the interactions between two types of brain cells: pyramidal cells and interneurons \cite{Wilson1972, Whittington2000, Hansel2003}. Pyramidal cells are excitatory neurons that transmit signals to other brain regions, while interneurons are inhibitory neurons that help regulate the activity of pyramidal cells. The PING model focuses on the transmission of signals between these neurons. In contrast, the Kuramoto model focuses solely on the synchronization of oscillating units \cite{Breakspear2010}, such as the firing of neurons. This difference in focus means that the PING model is considered to be more biologically plausible, as it more accurately represents the complex interactions between neurons in the brain.

The present thesis uses the PING model to investigate the role of gamma oscillations in \stimfig-ground segregation. We evaluate the extent to which synchrony of gamma oscillations, as modeled by the PING network, is predictive of human performance in experiments conducted by Karimian \cite{MaryamPLACEHOLDER}. 
We imitate the behavior of neurons using the Izhikevich model, which has been shown to exhibit a wide range of computational properties, including the ability to generate gamma-band oscillations \cite{Izhikevich2003}. To use as an input to the model, we reconstruct the stimuli introduced by Karimian \cite{MaryamPLACEHOLDER}. We then present the results of neural synchronization in our own simulations. The results show that synchrony of gamma oscillations is predictive of human performance in \stimfig-ground segregation, consistent with previous research.

\todo{write about thesis organization}




\section{Neural model}
\label{sec:grid-ping}

The stimulation of neural circuits, such as the induction of electrical transmembrane currents, can cause changes in the electrical potential across the membranes of neurons \cite{IzhikevichBook2004:2}. These changes can result in fluctuations in neural membrane potential. 
When the input currents surpass a certain threshold, they can cause a sudden and temporary change in the membrane potential that can be transmitted to other neurons: postsynaptic neurons receiving stimulation from presynaptic neurons fire an action potential if the summation of their presynaptic input surpasses a threshold, as discussed in Section \ref{sec:action-potential}. In addition, the presynaptic input can be inhibitory or excitatory, depending on the type of presynaptic neurons. Inhibitory neurons release neurotransmitters that reduce the probability of, or inhibit, the firing of an action potential in a postsynaptic neuron; excitatory neurons release neurotransmitters that increase that likelihood. 

In the pyramidal interneuron network gamma (PING) model, the interplay between a group of pyramidal neurons and interneurons gives rise to synchronous rhythmic spikes in the gamma frequency range (25-80 Hz)  \cite{Whittington2000, Borgers2003}. Pyramidal neurons use the excitatory neurotransmitter AMPA to stimulate other neurons, while interneurons use the inhibitory neurotransmitter GABA to reduce the activity of neurons. The schematic architecture of the network is presented in Figure \ref{fig:single-ping}.

\begin{figure}[!htp]
    \centering
    \newcommand{\spexy}{1.5}


\newcommand{\spiny}{0}
\newcommand{\spidx}{-1.5}
\newcommand{\splx}{1.5}
\newcommand{\spltx}{2}

\begin{tikzpicture}[
    arr/.style = { -{Stealth[ ]} },
    cnodeex/.style = {circle,thick,fill=color-two},
    cnodeex2/.style = {circle,very thick,draw=color-two, fill=color-two, fill opacity=0.2},
    cnodein/.style = {circle,thick,fill=color-one},
    cnodein2/.style = {circle,very thick,draw=color-one,fill=color-one, fill opacity=0.2},
    inarrow/.style = {arr, draw=color-one, fill=color-one, ultra thick},
    exarrow/.style = {arr, draw=color-two, fill=color-one, ultra thick},
    idl/.style = {draw=color-three, fill=color-one, ultra thick},
]
    
    \begin{scope}
        % ex
        \node[cnodeex] (ex) at (0, \spexy) {};
        \node[cnodeex2, shift={(-0.25, 0.2)}] at (0, \spexy) {};
        \node[cnodeex2, shift={(0.2, 0.25)}] at (0, \spexy) {};
        
        % in
        \node[cnodein] (in) at (0, \spiny) {};
        \node[cnodein2, shift={(0.1, -0.2)}] (in1) at (0, \spiny) {};
        
        % arows
        \path[inarrow, shift={(-0.1, 0)}] (in) edge[bend left=15] node {} (ex);
        \path[exarrow, shift={(0.1, 0)}] (ex) edge[bend left=15] node {} (in);
        
        % input drive
        \node[rotate=90] at (\spidx - 0.4, 0.5 * \spexy)  {{\small input drive}};
        \draw[idl] (\spidx, \spiny) -- (\spidx, \spexy);
        \draw[arr, idl] (\spidx, \spiny) -- (-0.9, \spiny);
        \draw[arr, idl] (\spidx, \spexy) -- (-0.9, \spexy);
    \end{scope}
    
    \begin{scope}
        % ex cells
        \node[cnodeex, anchor=west] at (\splx, \spexy - 0 * \spexy) {};
        \node[anchor=west] at (\spltx, \spexy - 0 * \spexy) {ex cells (pyramidal neurons)};
        
        % in cells
        \node[cnodein, anchor=west] at (\splx, \spexy - 0.33 * \spexy) {};
        \node[anchor=west] at (\spltx, \spexy - 0.33 * \spexy) {in cells (interneurons)};
        
        % ex
        \draw[exarrow] (\splx, \spexy - 0.67 * \spexy) -- (\spltx, \spexy - 0.67 * \spexy);
        \node[anchor=west] at (\spltx, \spexy - 0.67 * \spexy) {excitation (AMPA)};
        
        % in
        \draw[inarrow] (\splx, \spexy - 1 * \spexy) -- (\spltx, \spexy - 1 * \spexy);
        \node[anchor=west] at (\spltx, \spexy - 1 * \spexy) {inhibition (GABA)};
    \end{scope}
\end{tikzpicture}
    \caption[Schematic architecture of PING]{Schematic architecture of the pyramidal-interneuron network gamma (PING) \cite{Lowet2015}.}
    \label{fig:single-ping}
\end{figure}

Specifically, in this network, gamma oscillations are instigated by the input to excitatory neurons, followed by inhibitory feedback \cite{Whittington2000}. When there is enough excitatory activity, pyramidal cells excite fast-spiking (FS) interneurons, which then become depolarized and inhibit the pyramidal neurons. The inhibition of the pyramidal cells stops the depolarization of the FS neurons, ending their inhibitory effect.
This creates a time interval for the pyramidal cells to fire again until they are inhibited by the FS cells. This creates a feedback loop that can generate sustained gamma oscillations in the network \cite{Kopell2011}.

\subsection{Current from the stimulus}

\subsubsection{From luminance to contrast}

\begin{figure}[t]
    \centering
    \begin{subfigure}[t]{0.4\textwidth}
        \centering
        \newcommand{\splw}{0.9\textwidth}
\newcommand{\splh}{\splw}

\newcommand{\splnrverticallines}{2}
\newcommand{\splverticalinterval}{0.33333} % 1 / (splnrverticallines + 1)

\newcommand{\splnrhorizontallines}{2}
\newcommand{\splhorizontalinterval}{0.33333}

\begin{tikzpicture}[
    cline/.style = {very thick, main-color},
]
    \begin{scope}
        \node[anchor=south west,inner sep=0] at (0,0) {\includegraphics[width=\splw]{assets/images/stimulus-patch.png}};
        
        % vertical
        \foreach \i in {1, 2, ..., \splnrverticallines}{
            \draw[cline] (\i * \splverticalinterval * \splw, 0) -- (\i * \splverticalinterval * \splw, \splh) {};
        }
        
        % horizontal
        \foreach \i in {1, 2, ..., \splnrverticallines}{
            \draw[cline] (0, \i * \splhorizontalinterval * \splh) -- (\splw, \i * \splhorizontalinterval * \splh) {};
        }

    \end{scope}
\end{tikzpicture}
        \caption{A 20 $\times$ 20 lattice applied to a patch.}
        \label{fig:llc-lattice-example}
    \end{subfigure}
    \hspace{0.06\textwidth}
    \begin{subfigure}[t]{0.4\textwidth}
        \centering
        \includegraphics[width=\textwidth]{src/assets/images/local-contrast.png}
        \caption{Local contrasts of the patch in Figure \ref{fig:llc-lattice-example}: $0$ (black) $\to$ $1$ (white).}
        \label{fig:llc-local-contrast-example}
    \end{subfigure}
    \caption[Patch lattice and local contrast]{The lattice of a patch and its vertices' local contrasts. 
    The patch was generated with the parameters $\anndistscale = 1.5$, $\contrange = 1$.}
    \label{fig:lattice-local-contrast-example}
\end{figure}

To transform the stimulus patch $\patchmatrix$ to the patch of PING networks in V1, we apply the lattice of size $n \times n$, such that each oscillatory network contains a field of $m \times m$ pixels. An example of the resulting lattice is displayed in Figure \ref{fig:llc-lattice-example}.

Let $\pingnets$ be the set containing the PING networks. Let $\pix: \pingnets \to (\mathbb{Z}_+)^{2 \times m^2}$ be a function mapping a network to the set of all $m^2$ pixel coordinates it contains.
For each network $\pingnet \in \pingnets$, a local contrast $\LC_\pingnet: V \to [0, 1]$ is the weighted root-mean-squared (RMS) contrast defined as follows \cite{Frazor2006}:
\begin{equation}
    \LC_\pingnet = \sqrt{
        \frac{
            \sum_{(i, j) \in \patchmatrix} \weight_{\pingnet, (i, j)} \frac{(\patchmatrix_{i, j} - \overline{\fullmatrix})^2}{\overline{\fullmatrix}^2}
        }{
            \sum_{(i, j) \in \patchmatrix} \weight_{\pingnet, (i, j)}
        }
    },
\end{equation}
\begin{comment}
\begin{equation}
    \LC_\pingnet = \sqrt{
        \frac{
            \sum_{(i, j) \in \pix(\pingnet)} \weight_{\pingnet, (i, j)} \frac{(\patchmatrix_{i, j} - \overline{\fullmatrix})^2}{\overline{\fullmatrix}^2}
        }{
            \sum_{(i, j) \in \pix(\pingnet)} \weight_{\pingnet, (i, j)}
        }
    },
\end{equation}
\end{comment}
where $\overline{\fullmatrix}$ is the mean over all luminance values in $\fullmatrix$, and $\weight: \pingnets, \mathbb{N}^2 \to \mathbb{R}$ is the weight of a pixel with respect to a PING network. An example of a local contrast matrix can be seen in Figure \ref{fig:llc-local-contrast-example}.

The value of weight is specific to each network as it reflects its unique receptive field \cite{MaryamPLACEHOLDER}. Let $\centr : \pingnets \to \mathbb{R}_+^2$ be the function mapping a particular PING network in the patch to the location of its center. For the pixel $i, j \in \{ 0, 1, \cdots, \patchsize - 1 \}$, the weight with respect to a PING network $\pingnet$ is then defined as
\begin{equation}
    \weight_{\pingnet, (i, j)} = \exp \left(
        -\frac{\left( \frac{1}{\atopix} \cdot \| (i, j) - \centr(\pingnet) \|\right)^2}{ 2 \stdrf_\pingnet^2 }
    \right),
    \label{eq:weight-pixel-network}
\end{equation}
where $\stdrf_\pingnet: \pingnets \to \mathbb{R}$ is 
the standard deviation of the network's receptive field diameter. It has been found to be defined as follows \cite{Freeman2011, MaryamPLACEHOLDER}:  
\begin{equation}
    \stdrf_\pingnet = \frac{1}{4} \max{( \slopeRF \cdot \ecc_\pingnet + \interceptRF, \ \diamRF_{\min} )},
\end{equation}
where $\slopeRF \in \mathbb{R}$ is the slope of the receptive field, or the ratio of receptive field diameter to eccentricity, $\interceptRF \in \mathbb{R}$ is the intercept, and $\diamRF_{\min} \in \mathbb{R}_+$ is the diameter of the smallest receptive field. The values for these parameters are presented in Table \ref{tab:params-rf}.

\begin{table}[!htp]
    \centering
    %>{\columncolor{table-color}}
\begin{tabular}{rcc}
\hline
\multicolumn{2}{c}{\textbf{Parameter}} &  \textbf{Value} 
\\ \hline
RF slope & \textbf{${\slopeRF}$} & 0.172
\\ 
RF intercept & \textbf{${\interceptRF}$} & -0.25
\\ 
Min RF diameter & \textbf{${\diamRF_{\min}}$} & 1
\\ \hline
\end{tabular}
    \caption{The values of the parameters used to determine the standard deviation of the receptive field diameter \cite{MaryamPLACEHOLDER}.}
    \label{tab:params-rf}
\end{table}

As the center of the gaze (fovea) coincides with the center of the stimulus, the eccentricity $\ecc: \pingnets \to \mathbb{R}_+$ in degrees for a network $v \in \pingnets$ is
\begin{equation}
\begin{gathered}
    \ecc_{\pingnet} = \frac{1}{\atopix} \cdot \left\| \stimcenter - (\patchstart + \centr(\pingnet)) \right\|, \\
    \stimcenter = \left( \frac{\fullwidth}{2}, \frac{\fullheight}{2} \right).
\end{gathered}
\end{equation}
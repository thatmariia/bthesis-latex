\subsection{The network}
\label{sec:grid-network}

The network considered in this thesis is an oscillatory network that represents a patch of neural networks in V1. Each PING network stands for an oscillator reflecting a neural circuit in V1, which interacts with other neural circuits in this area through lateral connections.
In this oscillatory network, a group of oscillators is arranged in a $n \times n$ grid similar to the network illustrated in Figure \ref{fig:oscillatory-grid-graph}.

\begin{figure}[!htp]
    \centering
    \begin{tikzpicture}[
    arr/.style = { -{Stealth[ ]} },
    cnodeex/.style = {circle,thick,fill=color-two},
    cnodein/.style = {circle,thick,fill=color-one},
    cframerounded/.style = {rounded corners=0.8cm, very thick, color-three},
]
        
    % vertices 1
    \begin{scope}
        \draw[cframerounded] (0.5, 0.7) rectangle (3.5, 2.5) {};
        
        \node[style=cnodeex] (Vex0) at (1, 2) {};
        \node[style=cnodeex] (Vex1) at (2, 2) {};
        \node[style=cnodeex] (Vex2) at (3, 2) {};
        
        \node[style=cnodein] (Vin3) at (1.5, 1) {};
        \node[style=cnodein] (Vin4) at (2.5, 1) {};
    \end{scope}
    
    % vertices 2
    \begin{scope}
        \draw[cframerounded] (4.5, 0.7) rectangle (7.5, 2.5) {};
        
        \node[style=cnodeex] (Vex5) at (5, 2) {};
        \node[style=cnodeex] (Vex6) at (6, 2) {};
        \node[style=cnodeex] (Vex7) at (7, 2) {};
        
        \node[style=cnodein] (Vin8) at (5.5, 1) {};
        \node[style=cnodein] (Vin9) at (6.5, 1) {};
    \end{scope}
    
    % vertices 3
    \begin{scope}
        \draw[cframerounded] (8.5, 0.7) rectangle (11.5, 2.5) {};
        
        \node[style=cnodeex] (Vex10) at (9, 2) {};
        \node[style=cnodeex] (Vex11) at (10, 2) {};
        \node[style=cnodeex] (Vex12) at (11, 2) {};
        
        \node[style=cnodein] (Vin13) at (9.5, 1) {};
        \node[style=cnodein] (Vin14) at (10.5, 1) {};
    \end{scope}
    
    % vertices 4
    \begin{scope}
        \draw[cframerounded] (0.5, 3.7) rectangle (3.5, 5.5) {};
        
        \node[style=cnodeex] (Vex15) at (1, 5) {};
        \node[style=cnodeex] (Vex16) at (2, 5) {};
        \node[style=cnodeex] (Vex17) at (3, 5) {};
        
        \node[style=cnodein] (Vin18) at (1.5, 4) {};
        \node[style=cnodein] (Vin19) at (2.5, 4) {};
    \end{scope}
    
    % vertices 5
    \begin{scope}
        \draw[cframerounded] (4.5, 3.7) rectangle (7.5, 5.5) {};
        
        \node[style=cnodeex] (Vex20) at (5, 5) {};
        \node[style=cnodeex] (Vex21) at (6, 5) {};
        \node[style=cnodeex] (Vex22) at (7, 5) {};
        
        \node[style=cnodein] (Vin23) at (5.5, 4) {};
        \node[style=cnodein] (Vin24) at (6.5, 4) {};
    \end{scope}
    
    % vertices 6
    \begin{scope}
        \draw[cframerounded] (8.5, 3.7) rectangle (11.5, 5.5) {};
        
        \node[style=cnodeex] (Vex25) at (9, 5) {};
        \node[style=cnodeex] (Vex26) at (10, 5) {};
        \node[style=cnodeex] (Vex27) at (11, 5) {};
        
        \node[style=cnodein] (Vin28) at (9.5, 4) {};
        \node[style=cnodein] (Vin29) at (10.5, 4) {};
    \end{scope}
    
    % vertices 7
    \begin{scope}
        \draw[cframerounded] (0.5, 6.7) rectangle (3.5, 8.5) {};
        
        \node[style=cnodeex] (Vex30) at (1, 8) {};
        \node[style=cnodeex] (Vex31) at (2, 8) {};
        \node[style=cnodeex] (Vex32) at (3, 8) {};
        
        \node[style=cnodein] (Vin33) at (1.5, 7) {};
        \node[style=cnodein] (Vin34) at (2.5, 7) {};
    \end{scope}
    
    % vertices 8
    \begin{scope}
        \draw[cframerounded] (4.5, 6.7) rectangle (7.5, 8.5) {};
        
        \node[style=cnodeex] (Vex35) at (5, 8) {};
        \node[style=cnodeex] (Vex36) at (6, 8) {};
        \node[style=cnodeex] (Vex37) at (7, 8) {};
        
        \node[style=cnodein] (Vin38) at (5.5, 7) {};
        \node[style=cnodein] (Vin39) at (6.5, 7) {};
    \end{scope}
    
    % vertices 9
    \begin{scope}
        \draw[cframerounded] (8.5, 6.7) rectangle (11.5, 8.5) {};
        
        \node[style=cnodeex] (Vex40) at (9, 8) {};
        \node[style=cnodeex] (Vex41) at (10, 8) {};
        \node[style=cnodeex] (Vex42) at (11, 8) {};
        
        \node[style=cnodein] (Vin43) at (9.5, 7) {};
        \node[style=cnodein] (Vin44) at (10.5, 7) {};
    \end{scope}
    
    % =======================================================
    
    \begin{scope}[>={Stealth[black]}]
        % vertical 
        
        % 1 to 4
        %\draw[<->] (2.1, 2.5) to (2.1, 3.7);
        \draw[<->] (1.9, 3.7) to (1.9, 2.5);
        
        % 4 to 7
        %\draw[<->] (2.1, 5.5) to (2.1, 6.7);
        \draw[<->] (1.9, 6.7) to (1.9, 5.5);
        
        % 2 to 5
        %\draw[<->] (6.1, 2.5) to (6.1, 3.7);
        \draw[<->] (5.9, 3.7) to (5.9, 2.5);
        
        % 5 to 8
        %\draw[<->] (6.1, 5.5) to (6.1, 6.7);
        \draw[<->] (5.9, 6.7) to (5.9, 5.5);
        
        % 3 to 6
        %\draw[<->] (10.1, 2.5) to (10.1, 3.7);
        \draw[<->] (9.9, 3.7) to (9.9, 2.5);
        
        % 6 to 9
        %\draw[<->] (10.1, 5.5) to (10.1, 6.7);
        \draw[<->] (9.9, 6.7) to (9.9, 5.5);
        
        % horizontal
        
        % 1 to 2
        %\draw[<->] (3.5, 1.7) to (4.5, 1.7);
        \draw[<->] (4.5, 1.5) to (3.5, 1.5);
        
        % 2 to 3
        %\draw[<->] (7.5, 1.7) to (8.5, 1.7);
        \draw[<->] (8.5, 1.5) to (7.5, 1.5);
        
        % 4 to 5
        %\draw[<->] (3.5, 4.7) to (4.5, 4.7);
        \draw[<->] (4.5, 4.5) to (3.5, 4.5);
        
        % 5 to 6
        %\draw[<->] (7.5, 4.7) to (8.5, 4.7);
        \draw[<->] (8.5, 4.5) to (7.5, 4.5);
        
        % 7 to 8
        %\draw[<->] (3.5, 7.7) to (4.5, 7.7);
        \draw[<->] (4.5, 7.5) to (3.5, 7.5);
        
        % 8 to 9
        %\draw[<->] (7.5, 7.7) to (8.5, 7.7);
        \draw[<->] (8.5, 7.5) to (7.5, 7.5);
    \end{scope}
    
    % edges 
    \begin{scope}[>={Stealth[black]}]
        \draw[<->] (Vex0) edge node {} (Vex1);
        \draw[<->] (Vex0) edge[bend left=25] node {} (Vex2);
        \draw[<->] (Vex0) edge node {} (Vin3);
        \draw[<->] (Vex0) edge node {} (Vin4);
        \draw[<->] (Vex1) edge node {} (Vex2);
        \draw[<->] (Vex1) edge node {} (Vin3);
        \draw[<->] (Vex1) edge node {} (Vin4);
        \draw[<->] (Vex2) edge node {} (Vin3);
        \draw[<->] (Vex2) edge node {} (Vin4);
        \draw[<->] (Vin3) edge node {} (Vin4);
        \draw[<->] (Vex5) edge node {} (Vex6);
        \draw[<->] (Vex5) edge[bend left=25] node {} (Vex7);
        \draw[<->] (Vex5) edge node {} (Vin8);
        \draw[<->] (Vex5) edge node {} (Vin9);
        \draw[<->] (Vex6) edge node {} (Vex7);
        \draw[<->] (Vex6) edge node {} (Vin8);
        \draw[<->] (Vex6) edge node {} (Vin9);
        \draw[<->] (Vex7) edge node {} (Vin8);
        \draw[<->] (Vex7) edge node {} (Vin9);
        \draw[<->] (Vin8) edge node {} (Vin9);
        \draw[<->] (Vex10) edge node {} (Vex11);
        \draw[<->] (Vex10) edge[bend left=25] node {} (Vex12);
        \draw[<->] (Vex10) edge node {} (Vin13);
        \draw[<->] (Vex10) edge node {} (Vin14);
        \draw[<->] (Vex11) edge node {} (Vex12);
        \draw[<->] (Vex11) edge node {} (Vin13);
        \draw[<->] (Vex11) edge node {} (Vin14);
        \draw[<->] (Vex12) edge node {} (Vin13);
        \draw[<->] (Vex12) edge node {} (Vin14);
        \draw[<->] (Vin13) edge node {} (Vin14);
        \draw[<->] (Vex15) edge node {} (Vex16);
        \draw[<->] (Vex15) edge[bend left=25] node {} (Vex17);
        \draw[<->] (Vex15) edge node {} (Vin18);
        \draw[<->] (Vex15) edge node {} (Vin19);
        \draw[<->] (Vex16) edge node {} (Vex17);
        \draw[<->] (Vex16) edge node {} (Vin18);
        \draw[<->] (Vex16) edge node {} (Vin19);
        \draw[<->] (Vex17) edge node {} (Vin18);
        \draw[<->] (Vex17) edge node {} (Vin19);
        \draw[<->] (Vin18) edge node {} (Vin19);
        \draw[<->] (Vex20) edge node {} (Vex21);
        \draw[<->] (Vex20) edge[bend left=25] node {} (Vex22);
        \draw[<->] (Vex20) edge node {} (Vin23);
        \draw[<->] (Vex20) edge node {} (Vin24);
        \draw[<->] (Vex21) edge node {} (Vex22);
        \draw[<->] (Vex21) edge node {} (Vin23);
        \draw[<->] (Vex21) edge node {} (Vin24);
        \draw[<->] (Vex22) edge node {} (Vin23);
        \draw[<->] (Vex22) edge node {} (Vin24);
        \draw[<->] (Vin23) edge node {} (Vin24);
        \draw[<->] (Vex25) edge node {} (Vex26);
        \draw[<->] (Vex25) edge[bend left=25] node {} (Vex27);
        \draw[<->] (Vex25) edge node {} (Vin28);
        \draw[<->] (Vex25) edge node {} (Vin29);
        \draw[<->] (Vex26) edge node {} (Vex27);
        \draw[<->] (Vex26) edge node {} (Vin28);
        \draw[<->] (Vex26) edge node {} (Vin29);
        \draw[<->] (Vex27) edge node {} (Vin28);
        \draw[<->] (Vex27) edge node {} (Vin29);
        \draw[<->] (Vin28) edge node {} (Vin29);
        \draw[<->] (Vex30) edge node {} (Vex31);
        \draw[<->] (Vex30) edge[bend left=25] node {} (Vex32);
        \draw[<->] (Vex30) edge node {} (Vin33);
        \draw[<->] (Vex30) edge node {} (Vin34);
        \draw[<->] (Vex31) edge node {} (Vex32);
        \draw[<->] (Vex31) edge node {} (Vin33);
        \draw[<->] (Vex31) edge node {} (Vin34);
        \draw[<->] (Vex32) edge node {} (Vin33);
        \draw[<->] (Vex32) edge node {} (Vin34);
        \draw[<->] (Vin33) edge node {} (Vin34);
        \draw[<->] (Vex35) edge node {} (Vex36);
        \draw[<->] (Vex35) edge[bend left=25] node {} (Vex37);
        \draw[<->] (Vex35) edge node {} (Vin38);
        \draw[<->] (Vex35) edge node {} (Vin39);
        \draw[<->] (Vex36) edge node {} (Vex37);
        \draw[<->] (Vex36) edge node {} (Vin38);
        \draw[<->] (Vex36) edge node {} (Vin39);
        \draw[<->] (Vex37) edge node {} (Vin38);
        \draw[<->] (Vex37) edge node {} (Vin39);
        \draw[<->] (Vin38) edge node {} (Vin39);
        \draw[<->] (Vex40) edge node {} (Vex41);
        \draw[<->] (Vex40) edge[bend left=25] node {} (Vex42);
        \draw[<->] (Vex40) edge node {} (Vin43);
        \draw[<->] (Vex40) edge node {} (Vin44);
        \draw[<->] (Vex41) edge node {} (Vex42);
        \draw[<->] (Vex41) edge node {} (Vin43);
        \draw[<->] (Vex41) edge node {} (Vin44);
        \draw[<->] (Vex42) edge node {} (Vin43);
        \draw[<->] (Vex42) edge node {} (Vin44);
        \draw[<->] (Vin43) edge node {} (Vin44);
    \end{scope}
\end{tikzpicture}
    \caption{An example of a 3 $\times$ 3 oscillatory network arranged in a grid. The arrows represent pairwise connections between all neurons (only shown for neighbouring cells, exist for all neurons). All connections in the network represent a bidirectional interaction.}
    \label{fig:oscillatory-grid-graph}
\end{figure}

This oscillatory network can be represented as a complete weighted directed graph $D = (V, E)$, where the set of vertices $V$ represents all neurons in the network, and the set of edges $E$ represents the synapses. Structurally, the oscillatory network connects $n^2$ identical subgraphs, each representing a PING network. Each subgraph contains the same number of excitatory and inhibitory neurons. An example of such a subgraph is visualized in Figure \ref{fig:single-ping-graph}. 
The cost function $\cost: E \to \mathbb{R}$ for the graph $D$ that reflects the interactions between neurons is defined later in Equation (\ref{eq:cost-function}).

\begin{figure}[!htp]
    \centering
    \begin{tikzpicture}[
    arr/.style = { -{Stealth[ ]} },
    cnodeex/.style = {circle,thick,fill=color-two-light},
    cnodein/.style = {circle,thick,fill=color-one-light},
    cframerounded/.style = {rounded corners=0.8cm, very thick, color-three},
]

    \begin{scope}
        \node (in) at (6, 6.6) {Excitatory neurons $V_\ex$};
        \draw[cframerounded] (0, 4.3) rectangle (12, 6.3) {};
        \node[style=cnodeex] (Vex0) at (1, 5) {$v_0$};
        \node[style=cnodeex] (Vex1) at (6, 5) {$v_1$};
        \node[style=cnodeex] (Vex2) at (11, 5) {$v_2$};
        
        \node (in) at (6, 0) {Inhibitory neurons $V_\inh$};
        \draw[cframerounded] (2.5, 0.5) rectangle (9.5, 2.1) {};
        \node[style=cnodein] (Vin3) at (3.5, 1.3) {$v_3$};
        \node[style=cnodein] (Vin4) at (8.5, 1.3) {$v_4$};
    \end{scope}
    
    \begin{scope}[>={Stealth[black]}]
        
        \path [->] (Vex0) edge[bend left=7] node {} (Vex1);
        \path [->] (Vex1) edge[bend left=7] node {} (Vex0);
        
        \path [->] (Vex0) edge[bend left=12] node {} (Vex2);
        \path [->] (Vex2) edge[bend right=18] node {} (Vex0);
        
        \path [->] (Vex1) edge[bend left=7] node {} (Vex2);
        \path [->] (Vex2) edge[bend left=7] node {} (Vex1);
        
        \path [->] (Vin3) edge[bend left=7] node {} (Vin4);
        \path [->] (Vin4) edge[bend left=7] node {} (Vin3);
        
        \path [->] (Vex0) edge[bend left=7] node {} (Vin3);
        \path [->] (Vin3) edge[bend left=7] node[below left] {$\cost_{(v_3 \to v_0)}$} (Vex0);
        
        \path [->] (Vex0) edge[bend left=7] node {} (Vin4);
        \path [->] (Vin4) edge[bend left=7] node {} (Vex0);
        
        \path [->] (Vex1) edge[bend left=7] node {} (Vin3);
        \path [->] (Vin3) edge[bend left=7] node {} (Vex1);
        
        \path [->] (Vex1) edge[bend left=7] node {} (Vin4);
        \path [->] (Vin4) edge[bend left=7] node {} (Vex1);
        
        \path [->] (Vex2) edge[bend left=7] node {} (Vin3);
        \path [->] (Vin3) edge[bend left=7] node {} (Vex2);
        
        \path [->] (Vex2) edge[bend left=7] node[below right] {$\cost_{(v_2 \to v_4)}$} (Vin4);
        \path [->] (Vin4) edge[bend left=7] node {} (Vex2);
    \end{scope}
\end{tikzpicture}
    \caption{An example of a PING network represented as graph with three excitatory and two inhibitory neurons. Each edge has a cost function but only two are displayed for aesthetic purposes.}
    \label{fig:single-ping-graph}
\end{figure}

\subsection{Current from the stimulus}

\subsubsection{From luminance to contrast}

\begin{figure}[t]
    \centering
    \begin{subfigure}[t]{0.4\textwidth}
        \centering
        \newcommand{\splw}{0.9\textwidth}
\newcommand{\splh}{\splw}

\newcommand{\splnrverticallines}{2}
\newcommand{\splverticalinterval}{0.33333} % 1 / (splnrverticallines + 1)

\newcommand{\splnrhorizontallines}{2}
\newcommand{\splhorizontalinterval}{0.33333}

\begin{tikzpicture}[
    cline/.style = {very thick, main-color},
]
    \begin{scope}
        \node[anchor=south west,inner sep=0] at (0,0) {\includegraphics[width=\splw]{assets/images/stimulus-patch.png}};
        
        % vertical
        \foreach \i in {1, 2, ..., \splnrverticallines}{
            \draw[cline] (\i * \splverticalinterval * \splw, 0) -- (\i * \splverticalinterval * \splw, \splh) {};
        }
        
        % horizontal
        \foreach \i in {1, 2, ..., \splnrverticallines}{
            \draw[cline] (0, \i * \splhorizontalinterval * \splh) -- (\splw, \i * \splhorizontalinterval * \splh) {};
        }

    \end{scope}
\end{tikzpicture}
        \caption{A 20 $\times$ 20 lattice applied to a patch.}
        \label{fig:llc-lattice-example}
    \end{subfigure}
    \hspace{0.06\textwidth}
    \begin{subfigure}[t]{0.4\textwidth}
        \centering
        \includegraphics[width=\textwidth]{src/assets/images/local-contrast.png}
        \caption{Local contrasts of the patch in Figure \ref{fig:llc-lattice-example}: $0$ (black) $\to$ $1$ (white).}
        \label{fig:llc-local-contrast-example}
    \end{subfigure}
    \caption[Patch lattice and local contrast]{The lattice of a patch and its vertices' local contrasts. 
    The patch was generated with the parameters $\anndistscale = 1.5$, $\contrange = 1$.}
    \label{fig:lattice-local-contrast-example}
\end{figure}

To transform the stimulus patch $\patchmatrix$ to the patch of PING networks in V1, we apply the lattice of size $n \times n$, such that each oscillatory network contains a field of $m \times m$ pixels. An example of the resulting lattice is displayed in Figure \ref{fig:llc-lattice-example}.

Let $\pingnets$ be the set containing the PING networks. Let $\pix: \pingnets \to (\mathbb{Z}_+)^{2 \times m^2}$ be a function mapping a network to the set of all $m^2$ pixel coordinates it contains.
For each network $\pingnet \in \pingnets$, a local contrast $\LC_\pingnet: V \to [0, 1]$ is the weighted root-mean-squared (RMS) contrast defined as follows \cite{Frazor2006}:
\begin{equation}
    \LC_\pingnet = \sqrt{
        \frac{
            \sum_{(i, j) \in \patchmatrix} \weight_{\pingnet, (i, j)} \frac{(\patchmatrix_{i, j} - \overline{\fullmatrix})^2}{\overline{\fullmatrix}^2}
        }{
            \sum_{(i, j) \in \patchmatrix} \weight_{\pingnet, (i, j)}
        }
    },
\end{equation}
\begin{comment}
\begin{equation}
    \LC_\pingnet = \sqrt{
        \frac{
            \sum_{(i, j) \in \pix(\pingnet)} \weight_{\pingnet, (i, j)} \frac{(\patchmatrix_{i, j} - \overline{\fullmatrix})^2}{\overline{\fullmatrix}^2}
        }{
            \sum_{(i, j) \in \pix(\pingnet)} \weight_{\pingnet, (i, j)}
        }
    },
\end{equation}
\end{comment}
where $\overline{\fullmatrix}$ is the mean over all luminance values in $\fullmatrix$, and $\weight: \pingnets, \mathbb{N}^2 \to \mathbb{R}$ is the weight of a pixel with respect to a PING network. An example of a local contrast matrix can be seen in Figure \ref{fig:llc-local-contrast-example}.

The value of weight is specific to each network as it reflects its unique receptive field \cite{MaryamPLACEHOLDER}. Let $\centr : \pingnets \to \mathbb{R}_+^2$ be the function mapping a particular PING network in the patch to the location of its center. For the pixel $i, j \in \{ 0, 1, \cdots, \patchsize - 1 \}$, the weight with respect to a PING network $\pingnet$ is then defined as
\begin{equation}
    \weight_{\pingnet, (i, j)} = \exp \left(
        -\frac{\left( \frac{1}{\atopix} \cdot \| (i, j) - \centr(\pingnet) \|\right)^2}{ 2 \stdrf_\pingnet^2 }
    \right),
    \label{eq:weight-pixel-network}
\end{equation}
where $\stdrf_\pingnet: \pingnets \to \mathbb{R}$ is 
the standard deviation of the network's receptive field diameter. It has been found to be defined as follows \cite{Freeman2011, MaryamPLACEHOLDER}:  
\begin{equation}
    \stdrf_\pingnet = \frac{1}{4} \max{( \slopeRF \cdot \ecc_\pingnet + \interceptRF, \ \diamRF_{\min} )},
\end{equation}
where $\slopeRF \in \mathbb{R}$ is the slope of the receptive field, or the ratio of receptive field diameter to eccentricity, $\interceptRF \in \mathbb{R}$ is the intercept, and $\diamRF_{\min} \in \mathbb{R}_+$ is the diameter of the smallest receptive field. The values for these parameters are presented in Table \ref{tab:params-rf}.

\begin{table}[!htp]
    \centering
    %>{\columncolor{table-color}}
\begin{tabular}{rcc}
\hline
\multicolumn{2}{c}{\textbf{Parameter}} &  \textbf{Value} 
\\ \hline
RF slope & \textbf{${\slopeRF}$} & 0.172
\\ 
RF intercept & \textbf{${\interceptRF}$} & -0.25
\\ 
Min RF diameter & \textbf{${\diamRF_{\min}}$} & 1
\\ \hline
\end{tabular}
    \caption{The values of the parameters used to determine the standard deviation of the receptive field diameter \cite{MaryamPLACEHOLDER}.}
    \label{tab:params-rf}
\end{table}

As the center of the gaze (fovea) coincides with the center of the stimulus, the eccentricity $\ecc: \pingnets \to \mathbb{R}_+$ in degrees for a network $v \in \pingnets$ is
\begin{equation}
\begin{gathered}
    \ecc_{\pingnet} = \frac{1}{\atopix} \cdot \left\| \stimcenter - (\patchstart + \centr(\pingnet)) \right\|, \\
    \stimcenter = \left( \frac{\fullwidth}{2}, \frac{\fullheight}{2} \right).
\end{gathered}
\end{equation}
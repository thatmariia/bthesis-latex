\subsection{Network architecture}
\label{sec:grid-network}

The network considered in this thesis is an oscillatory network that represents a patch of neural networks in the primary visual cortex (V1). Each PING network stands for an oscillator reflecting a neural circuit in V1, which interacts with other neural circuits in this area through lateral connections.
In this oscillatory network, a group of oscillators is arranged in a $n \times n$ grid similar to the network illustrated in Figure \ref{fig:oscillatory-grid-graph}.

\begin{figure}[!htp]
    \centering
    \begin{tikzpicture}[
    arr/.style = { -{Stealth[ ]} },
    cnodeex/.style = {circle,thick,fill=main-color},
    cnodein/.style = {circle,thick,fill=sec-color},
    cframerounded/.style = {rounded corners=0.8cm, very thick, third-color},
]
        
    % vertices 1
    \begin{scope}
        \draw[cframerounded] (0.5, 0.7) rectangle (3.5, 2.5) {};
        
        \node[style=cnodeex] (Vex0) at (1, 2) {};
        \node[style=cnodeex] (Vex1) at (2, 2) {};
        \node[style=cnodeex] (Vex2) at (3, 2) {};
        
        \node[style=cnodein] (Vin3) at (1.5, 1) {};
        \node[style=cnodein] (Vin4) at (2.5, 1) {};
    \end{scope}
    
    % vertices 2
    \begin{scope}
        \draw[cframerounded] (4.5, 0.7) rectangle (7.5, 2.5) {};
        
        \node[style=cnodeex] (Vex5) at (5, 2) {};
        \node[style=cnodeex] (Vex6) at (6, 2) {};
        \node[style=cnodeex] (Vex7) at (7, 2) {};
        
        \node[style=cnodein] (Vin8) at (5.5, 1) {};
        \node[style=cnodein] (Vin9) at (6.5, 1) {};
    \end{scope}
    
    % vertices 3
    \begin{scope}
        \draw[cframerounded] (8.5, 0.7) rectangle (11.5, 2.5) {};
        
        \node[style=cnodeex] (Vex10) at (9, 2) {};
        \node[style=cnodeex] (Vex11) at (10, 2) {};
        \node[style=cnodeex] (Vex12) at (11, 2) {};
        
        \node[style=cnodein] (Vin13) at (9.5, 1) {};
        \node[style=cnodein] (Vin14) at (10.5, 1) {};
    \end{scope}
    
    % vertices 4
    \begin{scope}
        \draw[cframerounded] (0.5, 3.7) rectangle (3.5, 5.5) {};
        
        \node[style=cnodeex] (Vex15) at (1, 5) {};
        \node[style=cnodeex] (Vex16) at (2, 5) {};
        \node[style=cnodeex] (Vex17) at (3, 5) {};
        
        \node[style=cnodein] (Vin18) at (1.5, 4) {};
        \node[style=cnodein] (Vin19) at (2.5, 4) {};
    \end{scope}
    
    % vertices 5
    \begin{scope}
        \draw[cframerounded] (4.5, 3.7) rectangle (7.5, 5.5) {};
        
        \node[style=cnodeex] (Vex20) at (5, 5) {};
        \node[style=cnodeex] (Vex21) at (6, 5) {};
        \node[style=cnodeex] (Vex22) at (7, 5) {};
        
        \node[style=cnodein] (Vin23) at (5.5, 4) {};
        \node[style=cnodein] (Vin24) at (6.5, 4) {};
    \end{scope}
    
    % vertices 6
    \begin{scope}
        \draw[cframerounded] (8.5, 3.7) rectangle (11.5, 5.5) {};
        
        \node[style=cnodeex] (Vex25) at (9, 5) {};
        \node[style=cnodeex] (Vex26) at (10, 5) {};
        \node[style=cnodeex] (Vex27) at (11, 5) {};
        
        \node[style=cnodein] (Vin28) at (9.5, 4) {};
        \node[style=cnodein] (Vin29) at (10.5, 4) {};
    \end{scope}
    
    % vertices 7
    \begin{scope}
        \draw[cframerounded] (0.5, 6.7) rectangle (3.5, 8.5) {};
        
        \node[style=cnodeex] (Vex30) at (1, 8) {};
        \node[style=cnodeex] (Vex31) at (2, 8) {};
        \node[style=cnodeex] (Vex32) at (3, 8) {};
        
        \node[style=cnodein] (Vin33) at (1.5, 7) {};
        \node[style=cnodein] (Vin34) at (2.5, 7) {};
    \end{scope}
    
    % vertices 8
    \begin{scope}
        \draw[cframerounded] (4.5, 6.7) rectangle (7.5, 8.5) {};
        
        \node[style=cnodeex] (Vex35) at (5, 8) {};
        \node[style=cnodeex] (Vex36) at (6, 8) {};
        \node[style=cnodeex] (Vex37) at (7, 8) {};
        
        \node[style=cnodein] (Vin38) at (5.5, 7) {};
        \node[style=cnodein] (Vin39) at (6.5, 7) {};
    \end{scope}
    
    % vertices 9
    \begin{scope}
        \draw[cframerounded] (8.5, 6.7) rectangle (11.5, 8.5) {};
        
        \node[style=cnodeex] (Vex40) at (9, 8) {};
        \node[style=cnodeex] (Vex41) at (10, 8) {};
        \node[style=cnodeex] (Vex42) at (11, 8) {};
        
        \node[style=cnodein] (Vin43) at (9.5, 7) {};
        \node[style=cnodein] (Vin44) at (10.5, 7) {};
    \end{scope}
    
    % =======================================================
    
    \begin{scope}[>={Stealth[black]}]
        % vertical 
        
        % 1 to 4
        \draw[->] (2.1, 2.5) to[bend right] (2.1, 3.7);
        \draw[->] (1.9, 3.7) to[bend right] (1.9, 2.5);
        
        % 4 to 7
        \draw[->] (2.1, 5.5) to[bend right] (2.1, 6.7);
        \draw[->] (1.9, 6.7) to[bend right] (1.9, 5.5);
        
        % 2 to 5
        \draw[->] (6.1, 2.5) to[bend right] (6.1, 3.7);
        \draw[->] (5.9, 3.7) to[bend right] (5.9, 2.5);
        
        % 5 to 8
        \draw[->] (6.1, 5.5) to[bend right] (6.1, 6.7);
        \draw[->] (5.9, 6.7) to[bend right] (5.9, 5.5);
        
        % 3 to 6
        \draw[->] (10.1, 2.5) to[bend right] (10.1, 3.7);
        \draw[->] (9.9, 3.7) to[bend right] (9.9, 2.5);
        
        % 6 to 9
        \draw[->] (10.1, 5.5) to[bend right] (10.1, 6.7);
        \draw[->] (9.9, 6.7) to[bend right] (9.9, 5.5);
        
        % horizontal
        
        % 1 to 2
        \draw[->] (3.5, 1.7) to[bend left] (4.5, 1.7);
        \draw[->] (4.5, 1.5) to[bend left] (3.5, 1.5);
        
        % 2 to 3
        \draw[->] (7.5, 1.7) to[bend left] (8.5, 1.7);
        \draw[->] (8.5, 1.5) to[bend left] (7.5, 1.5);
        
        % 4 to 5
        \draw[->] (3.5, 4.7) to[bend left] (4.5, 4.7);
        \draw[->] (4.5, 4.5) to[bend left] (3.5, 4.5);
        
        % 5 to 6
        \draw[->] (7.5, 4.7) to[bend left] (8.5, 4.7);
        \draw[->] (8.5, 4.5) to[bend left] (7.5, 4.5);
        
        % 7 to 8
        \draw[->] (3.5, 7.7) to[bend left] (4.5, 7.7);
        \draw[->] (4.5, 7.5) to[bend left] (3.5, 7.5);
        
        % 8 to 9
        \draw[->] (7.5, 7.7) to[bend left] (8.5, 7.7);
        \draw[->] (8.5, 7.5) to[bend left] (7.5, 7.5);
    \end{scope}
    
    % edges 
    \begin{scope}
        \path (Vex0) edge node {} (Vex1);
        \path (Vex0) edge[bend left=25] node {} (Vex2);
        \path (Vex0) edge node {} (Vin3);
        \path (Vex0) edge node {} (Vin4);
        \path (Vex1) edge node {} (Vex2);
        \path (Vex1) edge node {} (Vin3);
        \path (Vex1) edge node {} (Vin4);
        \path (Vex2) edge node {} (Vin3);
        \path (Vex2) edge node {} (Vin4);
        \path (Vin3) edge node {} (Vin4);
        \path (Vex5) edge node {} (Vex6);
        \path (Vex5) edge[bend left=25] node {} (Vex7);
        \path (Vex5) edge node {} (Vin8);
        \path (Vex5) edge node {} (Vin9);
        \path (Vex6) edge node {} (Vex7);
        \path (Vex6) edge node {} (Vin8);
        \path (Vex6) edge node {} (Vin9);
        \path (Vex7) edge node {} (Vin8);
        \path (Vex7) edge node {} (Vin9);
        \path (Vin8) edge node {} (Vin9);
        \path (Vex10) edge node {} (Vex11);
        \path (Vex10) edge[bend left=25] node {} (Vex12);
        \path (Vex10) edge node {} (Vin13);
        \path (Vex10) edge node {} (Vin14);
        \path (Vex11) edge node {} (Vex12);
        \path (Vex11) edge node {} (Vin13);
        \path (Vex11) edge node {} (Vin14);
        \path (Vex12) edge node {} (Vin13);
        \path (Vex12) edge node {} (Vin14);
        \path (Vin13) edge node {} (Vin14);
        \path (Vex15) edge node {} (Vex16);
        \path (Vex15) edge[bend left=25] node {} (Vex17);
        \path (Vex15) edge node {} (Vin18);
        \path (Vex15) edge node {} (Vin19);
        \path (Vex16) edge node {} (Vex17);
        \path (Vex16) edge node {} (Vin18);
        \path (Vex16) edge node {} (Vin19);
        \path (Vex17) edge node {} (Vin18);
        \path (Vex17) edge node {} (Vin19);
        \path (Vin18) edge node {} (Vin19);
        \path (Vex20) edge node {} (Vex21);
        \path (Vex20) edge[bend left=25] node {} (Vex22);
        \path (Vex20) edge node {} (Vin23);
        \path (Vex20) edge node {} (Vin24);
        \path (Vex21) edge node {} (Vex22);
        \path (Vex21) edge node {} (Vin23);
        \path (Vex21) edge node {} (Vin24);
        \path (Vex22) edge node {} (Vin23);
        \path (Vex22) edge node {} (Vin24);
        \path (Vin23) edge node {} (Vin24);
        \path (Vex25) edge node {} (Vex26);
        \path (Vex25) edge[bend left=25] node {} (Vex27);
        \path (Vex25) edge node {} (Vin28);
        \path (Vex25) edge node {} (Vin29);
        \path (Vex26) edge node {} (Vex27);
        \path (Vex26) edge node {} (Vin28);
        \path (Vex26) edge node {} (Vin29);
        \path (Vex27) edge node {} (Vin28);
        \path (Vex27) edge node {} (Vin29);
        \path (Vin28) edge node {} (Vin29);
        \path (Vex30) edge node {} (Vex31);
        \path (Vex30) edge[bend left=25] node {} (Vex32);
        \path (Vex30) edge node {} (Vin33);
        \path (Vex30) edge node {} (Vin34);
        \path (Vex31) edge node {} (Vex32);
        \path (Vex31) edge node {} (Vin33);
        \path (Vex31) edge node {} (Vin34);
        \path (Vex32) edge node {} (Vin33);
        \path (Vex32) edge node {} (Vin34);
        \path (Vin33) edge node {} (Vin34);
        \path (Vex35) edge node {} (Vex36);
        \path (Vex35) edge[bend left=25] node {} (Vex37);
        \path (Vex35) edge node {} (Vin38);
        \path (Vex35) edge node {} (Vin39);
        \path (Vex36) edge node {} (Vex37);
        \path (Vex36) edge node {} (Vin38);
        \path (Vex36) edge node {} (Vin39);
        \path (Vex37) edge node {} (Vin38);
        \path (Vex37) edge node {} (Vin39);
        \path (Vin38) edge node {} (Vin39);
        \path (Vex40) edge node {} (Vex41);
        \path (Vex40) edge[bend left=25] node {} (Vex42);
        \path (Vex40) edge node {} (Vin43);
        \path (Vex40) edge node {} (Vin44);
        \path (Vex41) edge node {} (Vex42);
        \path (Vex41) edge node {} (Vin43);
        \path (Vex41) edge node {} (Vin44);
        \path (Vex42) edge node {} (Vin43);
        \path (Vex42) edge node {} (Vin44);
        \path (Vin43) edge node {} (Vin44);
    \end{scope}
\end{tikzpicture}
    \caption[Grid oscillatory network]{An example of a 3 $\times$ 3 oscillatory network arranged in a grid. The arrows represent pairwise connections between all neurons (only shown for neighbouring cells, exist for all neurons). All connections in the network represent a bidirectional interaction.}
    \label{fig:oscillatory-grid-graph}
\end{figure}

This oscillatory network can be represented as a complete weighted directed graph $D = (V, E)$, where the set of vertices $V$ represents all neurons in the network, and the set of edges $E$ represents the synapses. Structurally, the oscillatory network connects $n^2$ identical subgraphs, each representing a PING network. Each PING network must contain at least one excitatory and one inhibitory neuron. Thus, $|V| > 2 \cdot n^2$ must hold.

The set of neurons can be separated in the following way: $V = \{ V_{\ex}, V_{\inh}\}$, where $V_{\ex}$ is a set of excitatory neurons, and $V_{\inh}$ - a set of inhibitory neurons, such that $V_{\ex} \cap V_{\inh} = \emptyset$.
Therefore, we can define $\type: V \to \{ \ex, \ \inh \}$ to be a function mapping a neuron to its type: $\ex$ corresponds to excitatory, and $\inh$ - to inhibitory neurons. Each subgraph contains the same number of excitatory and inhibitory neurons. Thus, it must hold that
\begin{equation}
    (|V_\ex| \ | \ n^2) \land (|V_\inh| \ | \ n^2).
\end{equation}
An example of a mentioned subgraph is visualized in Figure \ref{fig:single-ping-graph}. 
The cost function $\cost: E \to \mathbb{R}$ for the graph $D$ that reflects the interactions between neurons is defined later in Equation (\ref{eq:cost-function}).

\begin{figure}[!htp]
    \centering
    \begin{tikzpicture}[
    arr/.style = { -{Stealth[ ]} },
    cnodeex/.style = {circle,thick,fill=main-color},
    cnodein/.style = {circle,thick,fill=sec-color},
    cframerounded/.style = {rounded corners=0.8cm, very thick, third-color},
]

    \begin{scope}
        \node (in) at (6, 6.6) {Excitatory neurons};
        \draw[cframerounded] (0, 4.3) rectangle (12, 6.3) {};
        \node[style=cnodeex] (Vex0) at (1, 5) {$v_0$};
        \node[style=cnodeex] (Vex1) at (6, 5) {$v_1$};
        \node[style=cnodeex] (Vex2) at (11, 5) {$v_2$};
        
        \node (in) at (6, 0) {Inhibitory neurons};
        \draw[cframerounded] (2.5, 0.5) rectangle (9.5, 2.1) {};
        \node[style=cnodein] (Vin3) at (3.5, 1.3) {$v_3$};
        \node[style=cnodein] (Vin4) at (8.5, 1.3) {$v_4$};
    \end{scope}
    
    \begin{scope}[>={Stealth[black]}]
        
        \path [->] (Vex0) edge[bend left=7] node {} (Vex1);
        \path [->] (Vex1) edge[bend left=7] node {} (Vex0);
        
        \path [->] (Vex0) edge[bend left=12] node {} (Vex2);
        \path [->] (Vex2) edge[bend right=18] node {} (Vex0);
        
        \path [->] (Vex1) edge[bend left=7] node {} (Vex2);
        \path [->] (Vex2) edge[bend left=7] node {} (Vex1);
        
        \path [->] (Vin3) edge[bend left=7] node {} (Vin4);
        \path [->] (Vin4) edge[bend left=7] node {} (Vin3);
        
        \path [->] (Vex0) edge[bend left=7] node {} (Vin3);
        \path [->] (Vin3) edge[bend left=7] node[below left] {$\cost_{(v_3 \to v_0)}$} (Vex0);
        
        \path [->] (Vex0) edge[bend left=7] node {} (Vin4);
        \path [->] (Vin4) edge[bend left=7] node {} (Vex0);
        
        \path [->] (Vex1) edge[bend left=7] node {} (Vin3);
        \path [->] (Vin3) edge[bend left=7] node {} (Vex1);
        
        \path [->] (Vex1) edge[bend left=7] node {} (Vin4);
        \path [->] (Vin4) edge[bend left=7] node {} (Vex1);
        
        \path [->] (Vex2) edge[bend left=7] node {} (Vin3);
        \path [->] (Vin3) edge[bend left=7] node {} (Vex2);
        
        \path [->] (Vex2) edge[bend left=7] node[below right] {$\cost_{(v_2 \to v_4)}$} (Vin4);
        \path [->] (Vin4) edge[bend left=7] node {} (Vex2);
    \end{scope}
\end{tikzpicture}
    \caption[PING network as a graph]{An example of a PING network represented as graph with three excitatory and two inhibitory neurons. Each edge has a cost function but only two are displayed for aesthetic reasons.}
    \label{fig:single-ping-graph}
\end{figure}

It has been experimentally found before that the ratio of the number of excitatory neurons ($|V_{\ex}|$) and the number of inhibitory neurons ($|V_{\inh}|$) in a cortical population is four to one \cite{Pastore2018}. While the relationship has been utilized in the present thesis, the realistic number of neurons has not as it is computationally infeasible. The chosen values for the number of neurons of each type and the number of PING networks are displayed in Table \ref{tab:params-network}.

\begin{table}[!htp]
    \centering
    \begin{tabular}{|
>{\columncolor{main-color}}c |c|}
\hline
\textbf{Parameter} & \cellcolor{main-color}{ \textbf{Value}} \\ \hline
\textbf{$\pmb{n}$}                        & 20                                                            \\ \hline
\textbf{$\pmb{|V_{\ex}|}$}                  & 19200                                                         \\ \hline
\textbf{$\pmb{|V_{\inh}|}$}                 & 4800                                                          \\ \hline
\end{tabular}
    \caption{Network parameters.}
    \label{tab:params-network}
\end{table}

In conclusion, we consider a large oscillatory network that consists of
\startbulletsnospace
\begin{itemize}
    \item $n \times n$ oscillatory networks (PING networks). Each of them consists of
    \vspace{-0.5em}
    \begin{itemize}
        \item $|V| / n^2$ oscillators (neurons):
        \begin{itemize}
            \item $|V_{\ex}| / n^2$ are excitatory, and
            \item $|V_{\inh}| / n^2$ - inhibitory.
        \end{itemize}
    \end{itemize}
\end{itemize}
\startbulletsnospace
Every single neuron is connected to and interacts with every other through synapses. 
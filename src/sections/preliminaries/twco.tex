\subsection{Theory of weakly coupled oscillators}
\label{sec:twco}

\todo{write about this}

\begin{notes}
    From Maryam's ch3 intro:

    A number of empirical and theoretical studies have shown that the precise gamma frequency elicited locally in retinotopic visual cortical areas such as V1 and V2 is determined by stimulus features such as contrast, local speed, and eccentricity. The effect of contrast on local gamma frequency has been particularly well-documented. This implies that two neighbouring neuronal groups stimulated by elements of similar contrast would show similar gamma frequencies (low detuning), whereas neighbouring neuronal groups stimulated by elements of dissimilar contrasts would show more distinct gamma frequencies (higher detuning). Coupling strength, the second parameter that affects synchrony, is directly related to anatomical connectivity among neural oscillators. This implies that neuronal populations in close proximity are more strongly coupled than neuronal populations that are further apart. In conjunction with the retinotopic organization of early visual cortex, these regions’ horizontal connectivity profile implies that neuronal populations encoding proximal elements in a texture stimulus are more strongly coupled than populations encoding distal elements.
\end{notes}
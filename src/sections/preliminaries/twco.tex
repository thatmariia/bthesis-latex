\subsection{Theory of weakly coupled oscillators}
\label{sec:twco}

The theory of weakly coupled oscillators (TWCO) is a mathematical framework that describes the behavior of systems made up of multiple interacting oscillators \cite{SchultheissBook2011:1}. An oscillator is any system that exhibits periodic behavior, such as a swinging pendulum. In the context of the TWCO, the term \say{weakly coupled} means that the interactions between the oscillators are relatively weak, indicating each oscillator behaves largely independently.

For example, a row of five pendulums hanging side by side on a long, flexible rod mounted on a wall would be considered weakly coupled. In this scenario, the pendulums would each exhibit their own periodic motion at their own frequency and continue to swing independently, but they may still be influenced by each other in some ways, such as transferring energy through the rod when one pendulum is pushed, leading to some level of temporary synchronization between the pendulums.

The synchronization of oscillators is the phenomenon where they begin to exhibit the same periodic behavior, which can occur when the oscillators are coupled. In the case of WCOs, the coupling is not strong enough for the oscillators to synchronize easily. Another important concept in the TWCO is the idea of phase synchronization, which is a more precise form of synchronization and can be more difficult to achieve in systems with weak coupling due to their high sensitivity towards noise. It can cause the oscillators to drift out of phase with each other over time, making it difficult to maintain phase synchronization.

The strength of the coupling determines how easily the oscillators synchronize with each other, and the difference between their frequencies (referred to as detuning) affects the stability of the synchronization \cite{Pikovsky2002, Tiesinga2010, Lowet2015}. When the frequencies of the oscillators are very similar, the synchronization is more stable and less sensitive to perturbations. The region in the coupling strength-detuning plane where the oscillators are able to synchronize is called \say{Arnold tongue}. As depicted in Figure \ref{fig:arnold-tongue-example}, this region is typically shaped like a tongue. The tip of the tongue represents the strongest coupling and smallest detuning required for synchronization, and the base of the tongue - the weakest coupling and largest detuning where synchronization is still possible. The boundaries of the Arnold tongue can also be affected by external factors, such as noise or perturbations, which can either enhance or disrupt the synchronization.

\begin{figure}[!htp]
    \centering
    % ----- INPUT
% 0.58267 * width
\newcommand{\arnoldexh}{0.2622\textwidth}
\newcommand{\arnoldexw}{0.45\textwidth}

% to shift for ticks
\newcommand{\arnoldexshiftx}{1.3em}
\newcommand{\arnoldexshifty}{1.3em}

% to shift to the middle from kinda the middle
\newcommand{\arnoldexmidshiftx}{0em}
\newcommand{\arnoldexmidshifty}{0em}

%
\newcommand{\arnoldexhormid}{0.5 * \arnoldexw}
\newcommand{\arnoldexvermid}{0.5 * \arnoldexh}



\begin{tikzpicture}[
        arr/.style = { -{Stealth[ ]} },
        bluearrow/.style = {arr, draw=third-color, fill=third-color, thick},
        blackarrow/.style = {arr, ultra thick},
    ]
    
    \begin{scope}

        \node[
            shift={(\arnoldexmidshiftx, -\arnoldexshifty)}
        ] at (\arnoldexhormid, 0) {{ \small Detuning} $\longrightarrow$ };
        
        \node[
            shift={(-\arnoldexshiftx, \arnoldexmidshifty)},
            rotate=90
        ] at (0, \arnoldexvermid) {{\small Coupling strength} $\longrightarrow$ };

        
        % \node[shift={(\apwest, \apshifth)}, anchor=west] at (0, \appeakh) {peak};
        
    \end{scope}
    
    \begin{scope}
        \node[anchor=south west,inner sep=0] at (0,0) {\includesquaregraphics[height=\arnoldexh]{src/assets/images/arnold-tongue-example.pdf}};
    \end{scope}
        
\end{tikzpicture}
    \caption[Arnold tongue shape]{The shape of the Arnold tongue.}
    \label{fig:arnold-tongue-example}
\end{figure}

In the context of neural synchronization, the TWCO can be used to model the interactions between neurons in a network and to predict the conditions under which they will synchronize. The notion of weak coupling is particularly relevant in this context, as it describes the relatively weak interactions between neurons in the network. 

Research has shown that the gamma frequency of brain activity in certain areas of the visual cortex, including V1, is determined by attributes of the stimuli, such as contrast, spatial frequency, speed, position in the visual field, stimulus size, and orientation \cite{Hubel1968, Kennedy1985, Gattass1987, Baldi1990, Whittington2000, Logothetis2001, Henrie2005, Hall2005, Buia2006, Gieselmann2008, Swettenham2009, Feng2010, Whittington2011, Roberts2013, Shapira2017, Dubey2020}. The effect of contrast on gamma frequency has been well-studied, and it has been found that neurons responding to similar contrast levels have similar gamma frequencies, or low detuning, while neurons responding to diverse contrast levels will have higher detuning \cite{Henrie2005, Ray2010, Roberts2013, Lowet2015, Hadjipapas2015, Shapira2017}.

The coupling strength of these neurons is influenced by their proximity to each other: neurons that are closer together are more strongly connected than neurons that are further apart \cite{Gilbert1983, Tso1986a, Stettler2002, Lowet2015, Lowet2017}. Together with the spatial relations in the retina (retinotopic organization), this suggests that the neuronal populations that encode nearby elements in a texture stimulus are coupled stronger than those that encode distant elements \cite{MaryamPLACEHOLDER}.
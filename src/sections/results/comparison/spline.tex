\subsubsection{Bivariate spline-interpolated data}

\paragraph{Statistical tests}

Similar to the original data, we have plotted the data (see Figure \ref{fig:relationship-spline}) and observed that the relationship between the two datasets $\simresspline$ and $\expresspline$ seems to be monotonic with positive coefficients. We then used Pearson's correlation coefficient to test the linearity of the relationship between the two variables. We found that it had a statistic of 0.6820 and a p-value of 0.0, indicating a strong linear relationship. We also performed a Kolmogorov-Smirnov normality test on the datasets and found that they had statistics of 0.6707 and 0.7395 with p-values of 0.0 for experimental and simulation data, respectively. Thus, neither dataset was normally distributed, and we are unable to draw conclusions about the significance of the linear relationship.
We then used Spearman's correlation and found that the relationship between the experimental and simulation results data was strongly monotonic and positive, with a statistic of 0.7496 and a p-value of 0.0. 

\paragraph{Regression analyses}

We further quantified the relationship between the bivariate splines of experimental and simulation data by fitting the linear, quadratic, and cubic polynomials. The resulting polynomials are shown in Table \ref{tab:results-summary} and visualized in Figure \ref{fig:relationship-fit-spline}.

The R2 goodness-of-fit scores for the resulting curves are 0.4648 (d1), 0.5007 (d2), and 0.5128 (d3), all of which are considered to be indicative of a moderate fit. The MSE CV scores in the same respective order are the following: 0.0012, 0.0011, and 0.0011. 

The summary of the statistical tests and the regression analyses performed on the bivariate spline data can be found in Table \ref{tab:results-summary}. Overall, greater Spearman's correlation, greater R2 scores, and lower CV values suggest that the correlation between the data interpolated with bivariate splines is stronger than between original data.

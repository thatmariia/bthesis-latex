\subsection{Experimental results}

In a recent study, Karimian performed a psychophysics experiment where participants were exposed to stimuli with the same set of parameters as used in the present thesis' simulations \cite{MaryamPLACEHOLDER}.
Eight participants with normal vision took part in this experiment. They were seated in a dimly lit room, viewing stimuli on a screen. They were asked to identify whether figures were positioned horizontally or vertically and press a corresponding button. Each experiment consisted of 30 blocks of 25 trials, with random combinations of parameters, for a total of 750 trials. The sequence of events in each trial was as follows: 
\begin{enumerate}
    \item a gray screen was presented;
    \item a small bright turquoise disk (the fixation point) appeared and lasted until the participant fixated on it accurately;
    \item a stimulus appeared and lasted for at most 1000 ms, or until the participant lost fixation or provided a response;
    \item feedback appeared for 500 ms.
\end{enumerate}
Trials in which the participant's gaze fell outside the screen were terminated and repeated at random times during the experiment. If a participant performed the task correctly, the value of one was recorded. Otherwise, a zero was recorded.
The results averaged for each set of stimulus parameters $\expres \in [0, 1]^{5 \times 5}$ are presented in Figure \ref{fig:exp-res}. 

Similar to the simulation data, 
the behavioral results are interpolated with the bivariate spline $\expresspline \in [0, 1]^{1000 \times 1000}$ and with the sigmoid function $\expressigmoid \in [0, 1]^{1000 \times 1000}$. The interpolations are shown in Figure \ref{fig:exp-res-spline} and Figure \ref{fig:exp-res-sigmoid}, respectively. The obtained sigmoid parameters $\sigmoidparams$ for the experimental results are given in Table \ref{tab:sigmoid-params-exp}.

\begin{table}[!htp]
    \centering
    \begin{tabular}{cr}
\hline
\textbf{Parameter} &  \textbf{Value} 
\\ \hline
 $\sigmoidparams_0$ & 3.6581
\\ 
 $\sigmoidparams_1$ & 3.5298
 \\
 $\sigmoidparams_2$ & -5.0670
\\ \hline
\end{tabular}
    \caption[Sigmoid parameters for experimental results]{Sigmoid parameters for experimental results.}
    \label{tab:sigmoid-params-exp}
\end{table}

\newpage

\begin{multicols}{2}

\begin{figure}[H]
     \centering
    \begin{subfigure}[t]{0.41\textwidth}
        \centering
        % ----- INPUT
\newcommand{\resphw}{\textwidth}
% 0.7829 * width
\newcommand{\resphh}{0.7829\textwidth}

% to shift for ticks
% \newcommand{\resshiftx}{1.3em}
% \newcommand{\resshifty}{1em}
\newcommand{\resshiftx}{0.65em}
\newcommand{\resshifty}{0.7em}

% to shift to the middle from kinda the middle
\newcommand{\resmidshiftx}{-0.5em}
\newcommand{\resmidshifty}{0.4em}

%
\newcommand{\reshormid}{0.5 * \resphw}
\newcommand{\resvermid}{0.5 * \resphh}


\newcommand{\resplot}[1]{
\begin{tikzpicture}[
        arr/.style = { -{Stealth[ ]} },
        bluearrow/.style = {arr, draw=third-color, fill=third-color, thick},
        blackarrow/.style = {arr, ultra thick},
    ]
    
    \begin{scope}

        \node[
            shift={(\resmidshiftx, -\resshifty)}
        ] at (\reshormid, 0) {{ \footnotesize Contrast}};
        
        \node[
            shift={(-\resshiftx, \resmidshifty)},
            rotate=90
        ] at (0, \resvermid) {{\footnotesize Coarseness}};

        
        % \node[shift={(\apwest, \apshifth)}, anchor=west] at (0, \appeakh) {peak};
        
    \end{scope}
    
    \begin{scope}
        \node[anchor=south west,inner sep=0] at (0,0) {\includegraphics[height=\resphh]{#1}};
    \end{scope}
        
\end{tikzpicture}
}
\resplot{src/assets/images/results/sim-res.pdf}
        \vspace{-\baselineskip}
        \caption{Average results after simulations.}
        \label{fig:sim-res}
    \end{subfigure}
    \\ \vspace{\baselineskip}
    \begin{subfigure}[t]{0.41\textwidth}
        \centering
        % ----- INPUT
\newcommand{\resphw}{\textwidth}
% 0.7829 * width
\newcommand{\resphh}{0.7829\textwidth}

% to shift for ticks
% \newcommand{\resshiftx}{1.3em}
% \newcommand{\resshifty}{1em}
\newcommand{\resshiftx}{0.65em}
\newcommand{\resshifty}{0.7em}

% to shift to the middle from kinda the middle
\newcommand{\resmidshiftx}{-0.5em}
\newcommand{\resmidshifty}{0.4em}

%
\newcommand{\reshormid}{0.5 * \resphw}
\newcommand{\resvermid}{0.5 * \resphh}


\newcommand{\resplot}[1]{
\begin{tikzpicture}[
        arr/.style = { -{Stealth[ ]} },
        bluearrow/.style = {arr, draw=third-color, fill=third-color, thick},
        blackarrow/.style = {arr, ultra thick},
    ]
    
    \begin{scope}

        \node[
            shift={(\resmidshiftx, -\resshifty)}
        ] at (\reshormid, 0) {{ \footnotesize Contrast}};
        
        \node[
            shift={(-\resshiftx, \resmidshifty)},
            rotate=90
        ] at (0, \resvermid) {{\footnotesize Coarseness}};

        
        % \node[shift={(\apwest, \apshifth)}, anchor=west] at (0, \appeakh) {peak};
        
    \end{scope}
    
    \begin{scope}
        \node[anchor=south west,inner sep=0] at (0,0) {\includegraphics[height=\resphh]{#1}};
    \end{scope}
        
\end{tikzpicture}
}
\resplot{src/assets/images/results/sim-res-spline.pdf}
        \vspace{-\baselineskip}
        \caption{Average simulation results interpolated with a bivariate spline.}
        \label{fig:sim-res-spline}
    \end{subfigure}
    \\ \vspace{\baselineskip}
    \begin{subfigure}[t]{0.41\textwidth}
        \centering
        % ----- INPUT
\newcommand{\resphw}{\textwidth}
% 0.7829 * width
\newcommand{\resphh}{0.7829\textwidth}

% to shift for ticks
% \newcommand{\resshiftx}{1.3em}
% \newcommand{\resshifty}{1em}
\newcommand{\resshiftx}{0.65em}
\newcommand{\resshifty}{0.7em}

% to shift to the middle from kinda the middle
\newcommand{\resmidshiftx}{-0.5em}
\newcommand{\resmidshifty}{0.4em}

%
\newcommand{\reshormid}{0.5 * \resphw}
\newcommand{\resvermid}{0.5 * \resphh}


\newcommand{\resplot}[1]{
\begin{tikzpicture}[
        arr/.style = { -{Stealth[ ]} },
        bluearrow/.style = {arr, draw=third-color, fill=third-color, thick},
        blackarrow/.style = {arr, ultra thick},
    ]
    
    \begin{scope}

        \node[
            shift={(\resmidshiftx, -\resshifty)}
        ] at (\reshormid, 0) {{ \footnotesize Contrast}};
        
        \node[
            shift={(-\resshiftx, \resmidshifty)},
            rotate=90
        ] at (0, \resvermid) {{\footnotesize Coarseness}};

        
        % \node[shift={(\apwest, \apshifth)}, anchor=west] at (0, \appeakh) {peak};
        
    \end{scope}
    
    \begin{scope}
        \node[anchor=south west,inner sep=0] at (0,0) {\includegraphics[height=\resphh]{#1}};
    \end{scope}
        
\end{tikzpicture}
}
\resplot{src/assets/images/results/sim-res-sigmoid.pdf}
        \vspace{-\baselineskip}
        \caption{Average simulation results interpolated with a psychometric sigmoid.}
        \label{fig:sim-res-sigmoid}
    \end{subfigure}
    \caption[Simulation results]{Average results after simulations and their interpolations.}
    \label{fig:all-sim-res}
\end{figure}

\columnbreak

\begin{figure}[H]
     \centering
    \begin{subfigure}[t]{0.41\textwidth}
        \centering
        % ----- INPUT
\newcommand{\resphw}{\textwidth}
% 0.7829 * width
\newcommand{\resphh}{0.7829\textwidth}

% to shift for ticks
% \newcommand{\resshiftx}{1.3em}
% \newcommand{\resshifty}{1em}
\newcommand{\resshiftx}{0.65em}
\newcommand{\resshifty}{0.7em}

% to shift to the middle from kinda the middle
\newcommand{\resmidshiftx}{-0.5em}
\newcommand{\resmidshifty}{0.4em}

%
\newcommand{\reshormid}{0.5 * \resphw}
\newcommand{\resvermid}{0.5 * \resphh}


\newcommand{\resplot}[1]{
\begin{tikzpicture}[
        arr/.style = { -{Stealth[ ]} },
        bluearrow/.style = {arr, draw=third-color, fill=third-color, thick},
        blackarrow/.style = {arr, ultra thick},
    ]
    
    \begin{scope}

        \node[
            shift={(\resmidshiftx, -\resshifty)}
        ] at (\reshormid, 0) {{ \footnotesize Contrast}};
        
        \node[
            shift={(-\resshiftx, \resmidshifty)},
            rotate=90
        ] at (0, \resvermid) {{\footnotesize Coarseness}};

        
        % \node[shift={(\apwest, \apshifth)}, anchor=west] at (0, \appeakh) {peak};
        
    \end{scope}
    
    \begin{scope}
        \node[anchor=south west,inner sep=0] at (0,0) {\includegraphics[height=\resphh]{#1}};
    \end{scope}
        
\end{tikzpicture}
}
\resplot{src/assets/images/results/exp-res.pdf}
        \vspace{-\baselineskip}
        \caption{Average results from experiments \cite{MaryamPLACEHOLDER}.}
        \label{fig:exp-res}
    \end{subfigure}
    \\ \vspace{\baselineskip}
    \begin{subfigure}[t]{0.41\textwidth}
        \centering
        % ----- INPUT
\newcommand{\resphw}{\textwidth}
% 0.7829 * width
\newcommand{\resphh}{0.7829\textwidth}

% to shift for ticks
% \newcommand{\resshiftx}{1.3em}
% \newcommand{\resshifty}{1em}
\newcommand{\resshiftx}{0.65em}
\newcommand{\resshifty}{0.7em}

% to shift to the middle from kinda the middle
\newcommand{\resmidshiftx}{-0.5em}
\newcommand{\resmidshifty}{0.4em}

%
\newcommand{\reshormid}{0.5 * \resphw}
\newcommand{\resvermid}{0.5 * \resphh}


\newcommand{\resplot}[1]{
\begin{tikzpicture}[
        arr/.style = { -{Stealth[ ]} },
        bluearrow/.style = {arr, draw=third-color, fill=third-color, thick},
        blackarrow/.style = {arr, ultra thick},
    ]
    
    \begin{scope}

        \node[
            shift={(\resmidshiftx, -\resshifty)}
        ] at (\reshormid, 0) {{ \footnotesize Contrast}};
        
        \node[
            shift={(-\resshiftx, \resmidshifty)},
            rotate=90
        ] at (0, \resvermid) {{\footnotesize Coarseness}};

        
        % \node[shift={(\apwest, \apshifth)}, anchor=west] at (0, \appeakh) {peak};
        
    \end{scope}
    
    \begin{scope}
        \node[anchor=south west,inner sep=0] at (0,0) {\includegraphics[height=\resphh]{#1}};
    \end{scope}
        
\end{tikzpicture}
}
\resplot{src/assets/images/results/exp-res-spline.pdf}
        \vspace{-\baselineskip}
        \caption{Average experimental results interpolated with a bivariate spline.}
        \label{fig:exp-res-spline}
    \end{subfigure}
    \\ \vspace{\baselineskip}
    \begin{subfigure}[t]{0.41\textwidth}
        \centering
        % ----- INPUT
\newcommand{\resphw}{\textwidth}
% 0.7829 * width
\newcommand{\resphh}{0.7829\textwidth}

% to shift for ticks
% \newcommand{\resshiftx}{1.3em}
% \newcommand{\resshifty}{1em}
\newcommand{\resshiftx}{0.65em}
\newcommand{\resshifty}{0.7em}

% to shift to the middle from kinda the middle
\newcommand{\resmidshiftx}{-0.5em}
\newcommand{\resmidshifty}{0.4em}

%
\newcommand{\reshormid}{0.5 * \resphw}
\newcommand{\resvermid}{0.5 * \resphh}


\newcommand{\resplot}[1]{
\begin{tikzpicture}[
        arr/.style = { -{Stealth[ ]} },
        bluearrow/.style = {arr, draw=third-color, fill=third-color, thick},
        blackarrow/.style = {arr, ultra thick},
    ]
    
    \begin{scope}

        \node[
            shift={(\resmidshiftx, -\resshifty)}
        ] at (\reshormid, 0) {{ \footnotesize Contrast}};
        
        \node[
            shift={(-\resshiftx, \resmidshifty)},
            rotate=90
        ] at (0, \resvermid) {{\footnotesize Coarseness}};

        
        % \node[shift={(\apwest, \apshifth)}, anchor=west] at (0, \appeakh) {peak};
        
    \end{scope}
    
    \begin{scope}
        \node[anchor=south west,inner sep=0] at (0,0) {\includegraphics[height=\resphh]{#1}};
    \end{scope}
        
\end{tikzpicture}
}
\resplot{src/assets/images/results/exp-res-sigmoid.pdf}
        \vspace{-\baselineskip}
        \caption{Average experimental results interpolated with a psychometric sigmoid.}
        \label{fig:exp-res-sigmoid}
    \end{subfigure}
    \caption[Experimental results]{Average results from behavioral experiments and their interpolations.}
    \label{fig:all-exp-res}
\end{figure}

\end{multicols}

\newpage

\begin{multicols}{2}

\begin{figure}[H]
     \centering
    \begin{subfigure}[t]{0.41\textwidth}
        \centering
        % ----- INPUT
% 0.951 * height
\newcommand{\relatphw}{0.8273775\textwidth}
\newcommand{\relatphh}{0.87\textwidth}

% to shift for ticks
% \newcommand{\relatshiftx}{1.3em}
% \newcommand{\relatshifty}{1em}
\newcommand{\relatshiftx}{1em}
\newcommand{\relatshifty}{0.7em}

% to shift to the middle from kinda the middle
\newcommand{\relatmidshiftx}{0.3em}
\newcommand{\relatmidshifty}{0.7em}

%
\newcommand{\relathormid}{0.5 * \relatphw}
\newcommand{\relatvermid}{0.5 * \relatphh}


\newcommand{\relatplot}[3]{
\begin{tikzpicture}[
        arr/.style = { -{Stealth[ ]} },
        bluearrow/.style = {arr, draw=third-color, fill=third-color, thick},
        blackarrow/.style = {arr, ultra thick},
    ]
    
    \begin{scope}

        \node[
            shift={(\relatmidshiftx, -\relatshifty)}
        ] at (\relathormid, 0) {{ \footnotesize #2}};
        
        \node[
            shift={(-\relatshiftx, \relatmidshifty)},
            rotate=90
        ] at (0, \relatvermid) {{\footnotesize #3}};

        
        % \node[shift={(\apwest, \apshifth)}, anchor=west] at (0, \appeakh) {peak};
        
    \end{scope}
    
    \begin{scope}
        \node[anchor=south west,inner sep=0] at (0,0) {\includegraphics[height=\relatphh]{#1}};
    \end{scope}
        
\end{tikzpicture}
}
\relatplot
{src/assets/images/relationships/relationship-original.pdf}
{Simulated avg PL ($\simres$)}
{Experimental avg PL ($\expres$)}
        %\vspace{-\baselineskip}
        \caption{Relationship between original data.}
        \label{fig:relationship-original}
    \end{subfigure}
    \\ \vspace{\baselineskip}
    \begin{subfigure}[t]{0.41\textwidth}
        \centering
        % ----- INPUT
% 0.951 * height
\newcommand{\relatphw}{0.8273775\textwidth}
\newcommand{\relatphh}{0.87\textwidth}

% to shift for ticks
% \newcommand{\relatshiftx}{1.3em}
% \newcommand{\relatshifty}{1em}
\newcommand{\relatshiftx}{1em}
\newcommand{\relatshifty}{0.7em}

% to shift to the middle from kinda the middle
\newcommand{\relatmidshiftx}{0.3em}
\newcommand{\relatmidshifty}{0.7em}

%
\newcommand{\relathormid}{0.5 * \relatphw}
\newcommand{\relatvermid}{0.5 * \relatphh}


\newcommand{\relatplot}[3]{
\begin{tikzpicture}[
        arr/.style = { -{Stealth[ ]} },
        bluearrow/.style = {arr, draw=third-color, fill=third-color, thick},
        blackarrow/.style = {arr, ultra thick},
    ]
    
    \begin{scope}

        \node[
            shift={(\relatmidshiftx, -\relatshifty)}
        ] at (\relathormid, 0) {{ \footnotesize #2}};
        
        \node[
            shift={(-\relatshiftx, \relatmidshifty)},
            rotate=90
        ] at (0, \relatvermid) {{\footnotesize #3}};

        
        % \node[shift={(\apwest, \apshifth)}, anchor=west] at (0, \appeakh) {peak};
        
    \end{scope}
    
    \begin{scope}
        \node[anchor=south west,inner sep=0] at (0,0) {\includegraphics[height=\relatphh]{#1}};
    \end{scope}
        
\end{tikzpicture}
}
\relatplot
{src/assets/images/relationships/relationship-spline.pdf}
{$\simresspline$}
{$\expresspline$}
        %\vspace{-\baselineskip}
        \caption{Relationship between bivariate spline-interpolated data.}
        \label{fig:relationship-spline}
    \end{subfigure}
    \\ \vspace{\baselineskip}
    \begin{subfigure}[t]{0.41\textwidth}
        \centering
        % ----- INPUT
% 0.951 * height
\newcommand{\relatphw}{0.8273775\textwidth}
\newcommand{\relatphh}{0.87\textwidth}

% to shift for ticks
% \newcommand{\relatshiftx}{1.3em}
% \newcommand{\relatshifty}{1em}
\newcommand{\relatshiftx}{1em}
\newcommand{\relatshifty}{0.7em}

% to shift to the middle from kinda the middle
\newcommand{\relatmidshiftx}{0.3em}
\newcommand{\relatmidshifty}{0.7em}

%
\newcommand{\relathormid}{0.5 * \relatphw}
\newcommand{\relatvermid}{0.5 * \relatphh}


\newcommand{\relatplot}[3]{
\begin{tikzpicture}[
        arr/.style = { -{Stealth[ ]} },
        bluearrow/.style = {arr, draw=third-color, fill=third-color, thick},
        blackarrow/.style = {arr, ultra thick},
    ]
    
    \begin{scope}

        \node[
            shift={(\relatmidshiftx, -\relatshifty)}
        ] at (\relathormid, 0) {{ \footnotesize #2}};
        
        \node[
            shift={(-\relatshiftx, \relatmidshifty)},
            rotate=90
        ] at (0, \relatvermid) {{\footnotesize #3}};

        
        % \node[shift={(\apwest, \apshifth)}, anchor=west] at (0, \appeakh) {peak};
        
    \end{scope}
    
    \begin{scope}
        \node[anchor=south west,inner sep=0] at (0,0) {\includegraphics[height=\relatphh]{#1}};
    \end{scope}
        
\end{tikzpicture}
}
\relatplot
{src/assets/images/relationships/relationship-sigmoid.pdf}
{$\simressigmoid$}
{$\expressigmoid$}
        %\vspace{-\baselineskip}
        \caption{Relationship between psychometric sigmoid-interpolated data.}
        \label{fig:relationship-sigmoid}
    \end{subfigure}
    \caption[Simulations-experiments relationships]{Relationship between original and interpolated data from experiments and simulations.}
    \label{fig:all-relationships}
\end{figure}

\columnbreak

\begin{figure}[H]
     \centering
    \begin{subfigure}[t]{0.41\textwidth}
        \centering
        % ----- INPUT
% 0.951 * height
\newcommand{\relatphw}{0.8273775\textwidth}
\newcommand{\relatphh}{0.87\textwidth}

% to shift for ticks
% \newcommand{\relatshiftx}{1.3em}
% \newcommand{\relatshifty}{1em}
\newcommand{\relatshiftx}{1em}
\newcommand{\relatshifty}{0.7em}

% to shift to the middle from kinda the middle
\newcommand{\relatmidshiftx}{0.3em}
\newcommand{\relatmidshifty}{0.7em}

%
\newcommand{\relathormid}{0.5 * \relatphw}
\newcommand{\relatvermid}{0.5 * \relatphh}


\newcommand{\relatplot}[3]{
\begin{tikzpicture}[
        arr/.style = { -{Stealth[ ]} },
        bluearrow/.style = {arr, draw=third-color, fill=third-color, thick},
        blackarrow/.style = {arr, ultra thick},
    ]
    
    \begin{scope}

        \node[
            shift={(\relatmidshiftx, -\relatshifty)}
        ] at (\relathormid, 0) {{ \footnotesize #2}};
        
        \node[
            shift={(-\relatshiftx, \relatmidshifty)},
            rotate=90
        ] at (0, \relatvermid) {{\footnotesize #3}};

        
        % \node[shift={(\apwest, \apshifth)}, anchor=west] at (0, \appeakh) {peak};
        
    \end{scope}
    
    \begin{scope}
        \node[anchor=south west,inner sep=0] at (0,0) {\includegraphics[height=\relatphh]{#1}};
    \end{scope}
        
\end{tikzpicture}
}
\relatplot
{src/assets/images/relationships/relationship-fit-original.pdf}
{$\simres$}
{$\expres$}
        %\vspace{-\baselineskip}
        \caption{Fitted curves between original data.}
        \label{fig:relationship-fit-original}
    \end{subfigure}
    \\ \vspace{\baselineskip}
    \begin{subfigure}[t]{0.41\textwidth}
        \centering
        % ----- INPUT
% 0.951 * height
\newcommand{\relatphw}{0.8273775\textwidth}
\newcommand{\relatphh}{0.87\textwidth}

% to shift for ticks
% \newcommand{\relatshiftx}{1.3em}
% \newcommand{\relatshifty}{1em}
\newcommand{\relatshiftx}{1em}
\newcommand{\relatshifty}{0.7em}

% to shift to the middle from kinda the middle
\newcommand{\relatmidshiftx}{0.3em}
\newcommand{\relatmidshifty}{0.7em}

%
\newcommand{\relathormid}{0.5 * \relatphw}
\newcommand{\relatvermid}{0.5 * \relatphh}


\newcommand{\relatplot}[3]{
\begin{tikzpicture}[
        arr/.style = { -{Stealth[ ]} },
        bluearrow/.style = {arr, draw=third-color, fill=third-color, thick},
        blackarrow/.style = {arr, ultra thick},
    ]
    
    \begin{scope}

        \node[
            shift={(\relatmidshiftx, -\relatshifty)}
        ] at (\relathormid, 0) {{ \footnotesize #2}};
        
        \node[
            shift={(-\relatshiftx, \relatmidshifty)},
            rotate=90
        ] at (0, \relatvermid) {{\footnotesize #3}};

        
        % \node[shift={(\apwest, \apshifth)}, anchor=west] at (0, \appeakh) {peak};
        
    \end{scope}
    
    \begin{scope}
        \node[anchor=south west,inner sep=0] at (0,0) {\includegraphics[height=\relatphh]{#1}};
    \end{scope}
        
\end{tikzpicture}
}
\relatplot
{src/assets/images/relationships/relationship-fit-spline.pdf}
{$\simresspline$}
{$\expresspline$}
        %\vspace{-\baselineskip}
        \caption{Fitted curves between bivariate spline-interpolated data.}
        \label{fig:relationship-fit-spline}
    \end{subfigure}
    \\ \vspace{\baselineskip}
    \begin{subfigure}[t]{0.41\textwidth}
        \centering
        % ----- INPUT
% 0.951 * height
\newcommand{\relatphw}{0.8273775\textwidth}
\newcommand{\relatphh}{0.87\textwidth}

% to shift for ticks
% \newcommand{\relatshiftx}{1.3em}
% \newcommand{\relatshifty}{1em}
\newcommand{\relatshiftx}{1em}
\newcommand{\relatshifty}{0.7em}

% to shift to the middle from kinda the middle
\newcommand{\relatmidshiftx}{0.3em}
\newcommand{\relatmidshifty}{0.7em}

%
\newcommand{\relathormid}{0.5 * \relatphw}
\newcommand{\relatvermid}{0.5 * \relatphh}


\newcommand{\relatplot}[3]{
\begin{tikzpicture}[
        arr/.style = { -{Stealth[ ]} },
        bluearrow/.style = {arr, draw=third-color, fill=third-color, thick},
        blackarrow/.style = {arr, ultra thick},
    ]
    
    \begin{scope}

        \node[
            shift={(\relatmidshiftx, -\relatshifty)}
        ] at (\relathormid, 0) {{ \footnotesize #2}};
        
        \node[
            shift={(-\relatshiftx, \relatmidshifty)},
            rotate=90
        ] at (0, \relatvermid) {{\footnotesize #3}};

        
        % \node[shift={(\apwest, \apshifth)}, anchor=west] at (0, \appeakh) {peak};
        
    \end{scope}
    
    \begin{scope}
        \node[anchor=south west,inner sep=0] at (0,0) {\includegraphics[height=\relatphh]{#1}};
    \end{scope}
        
\end{tikzpicture}
}
\relatplot
{src/assets/images/relationships/relationship-fit-sigmoid.pdf}
{$\simressigmoid$}
{$\expressigmoid$}
        %\vspace{-\baselineskip}
        \caption{Fitted curves between psychometric sigmoid-interpolated data.}
        \label{fig:relationship-fit-sigmoid}
    \end{subfigure}
    \caption[Simulations-experiments relationships with fitted curves]{Fitted curves between original and interpolated data from experiments and simulations.}
    \label{fig:all-fit-relationships}
\end{figure}

\end{multicols}
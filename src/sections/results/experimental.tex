\subsection{Experimental results}

In a recent study, Karimian performed a psychophysics experiment where participants were exposed to stimuli with the same set of parameters as used in the simulations \cite{MaryamPLACEHOLDER}.
Eight participants with normal vision took part in this experiment. They were seated in a dimly lit room, viewing stimuli on a screen. They were asked to identify whether figures were positioned horizontally or vertically by pressing a corresponding button. Each experiment consisted of 30 blocks of 25 trials, with random combinations of parameters, for a total of 750 trials. The sequence of events in each trial was as follows: 
\begin{enumerate}
    \item a gray screen was presented;
    \item a small bright turquoise disk (the fixation point) appeared and lasted until the participant fixated on it accurately;
    \item a stimulus appeared and lasted for at most 1000 ms, or until the participant lost fixation or provided a response;
    \item feedback appeared for 500 ms.
\end{enumerate}
Trials in which the participant's gaze fell outside the screen were terminated and repeated at random times during the experiment. If a participant performed the task correctly, the value of one was recorded. Otherwise, a zero was recorded.
The results averaged for each set of stimulus parameters are presented in Figure XXX. 
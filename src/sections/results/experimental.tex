\subsection{Experimental results}

\paragraph{Setup}

Eight participants with normal or corrected-to-normal visual acuity took part in the experiment \cite{MaryamPLACEHOLDER}. They were placed in a dimly lit room, where a chin and head-rest supported their heads. The stimuli were presented on a $19''$ screen $37.5$ cm apart from their eyes. The participants were required to identify whether the figure is positioned horizontally or vertically by pressing a corresponding button: right or left, respectively. The middle finger of the right hand was used to indicate the horizontal orientation, the index finger - vertical.

Each experiment consisted of a sequence of $30$ blocks of $25$ trials with particular combinations of parameters ($750$ trials in total) were used in a random order, exactly once per block. The order of events in each trial was the following:
\begin{enumerate}
    \item a gray screen was presented for $100$ ms;
    
    \item a small bright turquoise disk of $0.2^\circ \times 0.2^\circ$ - the fixation point - appeared and lasted for at least $1000$ ms, until the initiation of accurate fixation (i.e., the deviation of $< 2^\circ$ from the fixation point);
    
    \item a stimulus appeared and lasted for at most $1000$ ms: it was terminated earlier if the participant
    \begin{itemize}
        \item lost the fixation or
        \item pressed a button, providing a response;
    \end{itemize}
    
    \item the feedback in the form of the color change of the fixation point (green - correct, red - incorrect) appeared and lasted for $500$ ms.
\end{enumerate}
Between consecutive trials, the grey screen was presented for $600$ ms. \todo{including the 100 ms of the grey screen at the beginning of the trial?} Whenever a participant's gaze fell outside of the screen during step 2, the corresponding trial was terminated. All aborted trials were repeated at randomly chosen times during the experiment.


\paragraph{Results}

\todo{write this}
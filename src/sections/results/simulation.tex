\subsection{Simulation results}

For the input to the neural network, 25 Gabor texture stimuli have been constructed, each with a unique combination of contrast and grid coarseness, as shown in Table \ref{tab:stimulus-composition-params}. Examples of these stimuli are displayed in Table \ref{tab:25patches}.
\begin{table}[!hpt]
    \centering
    \newcommand{\patcheswidth}{2.4cm}
\setlength\extrarowheight{5pt}  
\newcommand{\raiseboxamounttwentyfive}{0} %{-.5
% {\includegraphics[width=\patcheswidth]
\newcommand{\rbtf}[1]{
    %\raisebox{\patcheswidth*\real{-0.5}}{#1}
    \raisebox{-.4\totalheight}{#1}
}

\newcommand{\patchcontent}[1]{
\begin{tikzpicture}
    \begin{scope}
    %\draw[red,thick,dashed] (0,0) circle (1.28cm);
    \node[anchor=center,inner sep=0] at (0,0) {\includegraphics[width=\patcheswidth]{src/assets/images/25patches/#1.png}};
    
    \end{scope}
\end{tikzpicture}
}
\newcommand{\patchA}{
    \patchcontent{dist-1.0000___contrast-0.0100}
}
\newcommand{\patchB}{
    \patchcontent{dist-1.0000___contrast-0.2575}
}
\newcommand{\patchC}{
    \patchcontent{dist-1.0000___contrast-0.5050}
}
\newcommand{\patchD}{
    \patchcontent{dist-1.0000___contrast-0.7525}
}
\newcommand{\patchE}{
    \patchcontent{dist-1.0000___contrast-1.0000}
}
\newcommand{\patchF}{
    \patchcontent{dist-1.1250___contrast-0.0100}
}
\newcommand{\patchG}{
    \patchcontent{dist-1.1250___contrast-0.2575}
}
\newcommand{\patchH}{
    \patchcontent{dist-1.1250___contrast-0.5050}
}
\newcommand{\patchI}{
    \patchcontent{dist-1.1250___contrast-0.7525}
}
\newcommand{\patchJ}{
    \patchcontent{dist-1.1250___contrast-1.0000}
}
\newcommand{\patchK}{
    \patchcontent{dist-1.2500___contrast-0.0100}
}
\newcommand{\patchL}{
    \patchcontent{dist-1.2500___contrast-0.2575}
}
\newcommand{\patchM}{
    \patchcontent{dist-1.2500___contrast-0.5050}
}
\newcommand{\patchN}{
    \patchcontent{dist-1.2500___contrast-0.7525}
}
\newcommand{\patchO}{
    \patchcontent{dist-1.2500___contrast-1.0000}
}
\newcommand{\patchP}{
    \patchcontent{dist-1.3750___contrast-0.0100}
}
\newcommand{\patchQ}{
    \patchcontent{dist-1.3750___contrast-0.2575}
}
\newcommand{\patchR}{
    \patchcontent{dist-1.3750___contrast-0.5050}
}
\newcommand{\patchS}{
    \patchcontent{dist-1.3750___contrast-0.7525}
}
\newcommand{\patchT}{
    \patchcontent{dist-1.3750___contrast-1.0000}
}
\newcommand{\patchU}{
    \patchcontent{dist-1.5000___contrast-0.0100}
}
\newcommand{\patchV}{
    \patchcontent{dist-1.5000___contrast-0.2575}
}
\newcommand{\patchW}{
    \patchcontent{dist-1.5000___contrast-0.5050}
}
\newcommand{\patchX}{
    \patchcontent{dist-1.5000___contrast-0.7525}
}
\newcommand{\patchY}{
    \patchcontent{dist-1.5000___contrast-1.0000}
}
%m{2.2cm}m{2.2cm}m{2.2cm}m{2.2cm}m{2.2cm}
%m{2.4cm}m{2.4cm}m{2.4cm}m{2.4cm}m{2.4cm}
\newcommand{\cwtw}{0.15\textwidth}

\begin{tabular}{|m{0.4cm}m{0.4cm}m{\cwtw}m{\cwtw}m{\cwtw}m{\cwtw}m{\cwtw}}
 \cline{3-7}
 \multicolumn{1}{c}{} & 
 \multicolumn{1}{c}{} &  
 \multicolumn{5}{c}{
    {\bf Contrast $\pmb{\contrange}$}
 } \\
 \multicolumn{1}{c}{} & 
 \multicolumn{1}{c}{} & 
 \multicolumn{1}{c}{0.01} & 
 \multicolumn{1}{c}{0.2570} & 
 \multicolumn{1}{c}{0.5050} & 
 \multicolumn{1}{c}{0.7525} & 
 \multicolumn{1}{c}{1}
 \\ \cline{3-7}
 & \multicolumn{1}{c|}{} & & & & & 
 \\[-2ex]
 & \multicolumn{1}{c|}{\rotatebox[origin=c]{90}{1}}        
 & \patchA & \patchB & \patchC & \patchD & \patchE 
 \\
 & \multicolumn{1}{c|}{} & & & & & 
 \\[-2ex]
 & \multicolumn{1}{c|}{\rotatebox[origin=c]{90}{1.125}}   
 & \patchF & \patchG & \patchH & \patchI & \patchJ 
 \\
 & \multicolumn{1}{c|}{} & & & & & 
 \\[-2ex]
 & \multicolumn{1}{c|}{\rotatebox[origin=c]{90}{1.250}}  
 & \patchK & \patchL & \patchM & \patchN & \patchO 
 \\
 & \multicolumn{1}{c|}{} & & & & & 
 \\[-2ex]
 & \multicolumn{1}{c|}{\rotatebox[origin=c]{90}{1.375}}
 & \patchP & \patchQ & \patchR & \patchS & \patchT 
 \\
 & \multicolumn{1}{c|}{} & & & & & 
 \\[-2ex]
 \multirow{-10}{*}[2.7cm]{ 
    \rotatebox[origin=c]{90}{{\bf Distance scale $\pmb{\anndistscale}$}}
 }
 & \multicolumn{1}{c|}{\rotatebox[origin=c]{90}{1.5}}
 & \patchU & \patchV & \patchW & \patchX & \patchY 
 \\

 \end{tabular}
    \caption[All stimuli examples]{An example of stimuli patches with all possible combinations of contrast and coarseness.}
    \label{tab:25patches}
\end{table}

The network with the parameters introduced in previous sections was simulated using input currents from each stimulus, and the resulting phase-locking values were calculated. This process has been repeated five times, and the average phase-locking values for each set of stimulus parameters were then calculated. Let $\simres \in [0, 1]^{5 \times 5}$ be the resulting matrix of average phase-locking values.
The values of $\simres$ are shown in Figure \ref{fig:sim-res}. The results of individual trials can be found in Appendix \ref{app:all-sim-results}.

We have plotted the spike rasters for three simulations with three different stimuli to display the level of phase-locking for a characteristic neural activity. The first simulation is expected to have the highest level of synchronization, the second - a medium level of synchronization, and the third - the least synchronization. The first plot (Figure \ref{fig:raster-best}) shows a high level of synchronization with an average phase-locking value of 0.8627, the second plot (Figure \ref{fig:raster-mid}) has a medium level of synchronization with a value of 0.7684, and the third plot (Figure \ref{fig:raster-worst}) has the lowest level of synchronization with a value of 0.5745.

\begin{figure}[p]
    \centering
    \begin{subfigure}[b]{\textwidth}
        \centering
        % ----- INPUT
\newcommand{\fullrasterimagew}{0.95\textwidth}
% 0.280455 * width
\newcommand{\fullrasterimageh}{0.266432\textwidth}

% to shift for ticks
\newcommand{\fullrastershiftx}{1.3em}
\newcommand{\fullrastershifty}{1em}

% to shift to the middle from kinda the middle
\newcommand{\fullrastermidshiftx}{0.5em}
\newcommand{\fullrastermidshifty}{0.4em}

%
\newcommand{\fullrasterhormid}{0.5 * \fullrasterimagew}
\newcommand{\fullrastervermid}{0.5 * \fullrasterimageh}


\newcommand{\fullraster}[1]{
\begin{tikzpicture}[
        arr/.style = { -{Stealth[ ]} },
        bluearrow/.style = {arr, draw=third-color, fill=third-color, thick},
        blackarrow/.style = {arr, ultra thick},
    ]
    
    \begin{scope}

        \node[
            shift={(\fullrastermidshiftx, -\fullrastershifty)}
        ] at (\fullrasterhormid, 0) {{ \small Time}};
        
        \node[
            shift={(-\fullrastershiftx, \fullrastermidshifty)},
            rotate=90
        ] at (0, \fullrastervermid) {{\small Neuron ID}};

        
        % \node[shift={(\apwest, \apshifth)}, anchor=west] at (0, \appeakh) {peak};
        
    \end{scope}
    
    \begin{scope}
        \node[anchor=south west,inner sep=0] at (0,0) {\includegraphics[height=\fullrasterimageh]{#1}};
    \end{scope}
        
\end{tikzpicture}
}
\fullraster{src/assets/images/rasters/raster-best.pdf}

        \vspace{-\baselineskip}
        \caption{The spike raster plot after a simulation with input the contrast of 0.01 and coarseness of 1.0. The corresponding average phase-locking value is 0.8627.}
        \label{fig:raster-best}
    \end{subfigure}
    \\ \vspace{\baselineskip}
    \begin{subfigure}[b]{\textwidth}
        \centering
        % ----- INPUT
\newcommand{\fullrasterimagew}{0.95\textwidth}
% 0.280455 * width
\newcommand{\fullrasterimageh}{0.266432\textwidth}

% to shift for ticks
\newcommand{\fullrastershiftx}{1.3em}
\newcommand{\fullrastershifty}{1em}

% to shift to the middle from kinda the middle
\newcommand{\fullrastermidshiftx}{0.5em}
\newcommand{\fullrastermidshifty}{0.4em}

%
\newcommand{\fullrasterhormid}{0.5 * \fullrasterimagew}
\newcommand{\fullrastervermid}{0.5 * \fullrasterimageh}


\newcommand{\fullraster}[1]{
\begin{tikzpicture}[
        arr/.style = { -{Stealth[ ]} },
        bluearrow/.style = {arr, draw=third-color, fill=third-color, thick},
        blackarrow/.style = {arr, ultra thick},
    ]
    
    \begin{scope}

        \node[
            shift={(\fullrastermidshiftx, -\fullrastershifty)}
        ] at (\fullrasterhormid, 0) {{ \small Time}};
        
        \node[
            shift={(-\fullrastershiftx, \fullrastermidshifty)},
            rotate=90
        ] at (0, \fullrastervermid) {{\small Neuron ID}};

        
        % \node[shift={(\apwest, \apshifth)}, anchor=west] at (0, \appeakh) {peak};
        
    \end{scope}
    
    \begin{scope}
        \node[anchor=south west,inner sep=0] at (0,0) {\includegraphics[height=\fullrasterimageh]{#1}};
    \end{scope}
        
\end{tikzpicture}
}
\fullraster{src/assets/images/rasters/raster-mid.pdf}

        \vspace{-\baselineskip}
        \caption{The spike raster plot after a simulation with input the contrast of 0.505 and coarseness of 1.25. The corresponding average phase-locking value is 0.7684.}
        \label{fig:raster-mid}
    \end{subfigure}
    \\ \vspace{\baselineskip}
    \begin{subfigure}[b]{\textwidth}
        \centering
        % ----- INPUT
\newcommand{\fullrasterimagew}{0.95\textwidth}
% 0.280455 * width
\newcommand{\fullrasterimageh}{0.266432\textwidth}

% to shift for ticks
\newcommand{\fullrastershiftx}{1.3em}
\newcommand{\fullrastershifty}{1em}

% to shift to the middle from kinda the middle
\newcommand{\fullrastermidshiftx}{0.5em}
\newcommand{\fullrastermidshifty}{0.4em}

%
\newcommand{\fullrasterhormid}{0.5 * \fullrasterimagew}
\newcommand{\fullrastervermid}{0.5 * \fullrasterimageh}


\newcommand{\fullraster}[1]{
\begin{tikzpicture}[
        arr/.style = { -{Stealth[ ]} },
        bluearrow/.style = {arr, draw=third-color, fill=third-color, thick},
        blackarrow/.style = {arr, ultra thick},
    ]
    
    \begin{scope}

        \node[
            shift={(\fullrastermidshiftx, -\fullrastershifty)}
        ] at (\fullrasterhormid, 0) {{ \small Time}};
        
        \node[
            shift={(-\fullrastershiftx, \fullrastermidshifty)},
            rotate=90
        ] at (0, \fullrastervermid) {{\small Neuron ID}};

        
        % \node[shift={(\apwest, \apshifth)}, anchor=west] at (0, \appeakh) {peak};
        
    \end{scope}
    
    \begin{scope}
        \node[anchor=south west,inner sep=0] at (0,0) {\includegraphics[height=\fullrasterimageh]{#1}};
    \end{scope}
        
\end{tikzpicture}
}
\fullraster{src/assets/images/rasters/raster-worst.pdf}

        \vspace{-\baselineskip}
        \caption{The spike raster plot after a simulation with input the contrast of 1.0 and coarseness of 1.5. The corresponding average phase-locking value is 0.5745.}
        \label{fig:raster-worst}
    \end{subfigure}
    \caption[Best-medium-worst spike trains]{Spike trains corresponding to expected best, medium, and worst average phase-locking values.}
    \label{fig:three-rasters}
\end{figure}

\paragraph{2D psychometric interpolation}

Psychometric functions are equations that show how someone's performance on a task relates to the difficulty of the stimulus or response. A sigmoid function is a common choice for this because it can model how a person's performance changes as the stimulus becomes more complex. The sigmoid we used to interpolate the data is the following:
\begin{equation}
    \sigmoidint(\anndistscale, \contrange, \sigmoidparams) = 0.5 \cdot \frac{1}{
    \exp(\sigmoidparams_0 \cdot \anndistscale + \sigmoidparams_1 \cdot \contrange + \sigmoidparams_2)
    } + 0.5.
\end{equation}

The parameter $\sigmoidparams = (\sigmoidparams_0, \sigmoidparams_1, \sigmoidparams_2) \in \mathbb{R}^3$ defines the shape of the sigmoid, and it is determined by fitting the psychometric function to the set of observed data. 

The curve fitting is performed using the least squares method with gradient descent (LSGD). It involves iteratively finding the values of the parameters in the vector $\sigmoidparams$ that minimize the sum of squares of residuals, or the loss function:
\begin{equation}
    \lossfunc(\simres, \sigmoidint, \sigmoidparams) = \sum_{\anndistscale, \contrange \in \arg (\simres)} 
    \left(
        \simres(\anndistscale, \contrange) - \sigmoidint(\anndistscale, \contrange, \sigmoidparams)
    \right)^2,
\end{equation}
where $\arg (\simres)$ is the set of all arguments of the matrix $\simres$.

\begin{table}[b]
    \centering
    \begin{tabular}{cr}
\hline
\textbf{Parameter} &  \textbf{Value} 
\\ \hline
 $\sigmoidparams_0$ & 0.7975
\\ 
 $\sigmoidparams_1$ & 2.1325
 \\
 $\sigmoidparams_2$ & -3.4625
\\ \hline
\end{tabular}
    \caption[Sigmoid parameters for simulation results]{Sigmoid parameters for simulation results.}
    \label{tab:sigmoid-params-sim}
\end{table}

At each iteration $i$, the algorithm uses the gradient of the loss function to determine the direction in which the parameters should be updated:
\begin{equation}
    \sigmoidparams(i) = \sigmoidparams(i-1) - \learningrate(i)
    \frac{
        \partial \lossfunc(\simres, \sigmoidint, \sigmoidparams(i-1))
    }{
        \partial {\sigmoidparams}
    },
\end{equation}
where $\learningrate = (\learningrate_0, \learningrate_1, \learningrate_2) \in \mathbb{R}^3$ is the learning rate, defined at each iteration by the Trust Region Reflective (TRF) method \cite{Branch1999}. It is determined by solving a trust-region subproblem, which involves finding the optimal step size that minimizes the loss function while staying within a specified trust region. The method is implemented in the $\mathsf{least\_squares}$ function from the $\mathsf{SciPy.optimize}$ package \cite{SciPy:least_squares}. 

The iterative process of adjusting the parameters continues until the optimal solution is achieved. The solution is considered optimal when at least one of the following criteria is met: the change in the loss function, the change in the parameters, or the norm of the gradient is less than 1e-8.

The values of $\sigmoidparams$ for the simulation result are shown in Table \ref{tab:sigmoid-params-sim}.
The simulation results interpolated with the 2D sigmoid psychometric function ($\simressigmoid \in [0, 1]^{1000 \times 1000}$) are shown in Figure \ref{fig:sim-res-sigmoid}.

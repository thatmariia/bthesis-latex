\subsection{Cortical coordinates}
\label{sec:cortical-coords}

As shown in Figure \ref{fig:receptive-field}, the receptive field varies in size with distance to the fovea and occupies a three-dimensional space. So, a two-dimensional stimulus projected onto the receptive field gets distorted. Therefore, a non-linear mapping is required to predict its location in the cortex. 

Let $\cmap: \mathbb{R}^2 \to \mathbb{R}^2$ be the function mapping the coordinates of a grid element (PING network) from the visual field to the visual cortex.
Let $\eccangle: V \to \left[ 0, \frac{\pi}{2} \right]$ be the angle between the horizontal axis and the line passing through points $\stimcenter$ and $(\patchstart + \centr(v))$. 
The complex-logarithm transformation from the cortical coordinates to the visual was introduced by Shwartz \cite{Schwartz1980}. The inverse of said transformation is used to define $\cmap$:
\begin{equation}
\begin{gathered}
    Z_v = \ecc_v \cdot \exp(i \cdot \cortalpha \cdot \eccangle_v). 
    \\
    W_v = \cortk \cdot \log \left( 
        \frac{Z_v + \corta}{Z_v + \cortb}
    \right) - \cortk \cdot \log \left( 
        \frac{\corta}{\cortb}
    \right).
    \\
    \cmap(v) = (\Re(W_v), \Im(W_v)).
\end{gathered}
\end{equation}
The values of parameters $\cortk, \corta, \cortb, \cortalpha$ are presented in Table \ref{tab:cortical-coords-params}.


\begin{table}[!htp]
    \centering
    \begin{tabular}{|
>{\columncolor{table-color}}c |c|c|}
\hline
\textbf{Parameter} & {\cellcolor{table-color}\textbf{Value}}  \\ \hline
$\pmb{\cortk}$ & $15$ \\ \hline
$\pmb{\corta}$ & $0.7$ \\ \hline
$\pmb{\cortb}$ & $80$ \\ \hline
$\pmb{\cortalpha}$ & $0.9$ \\ \hline
\end{tabular}
    \caption{Parameters for coordinates conversion \cite{Polimeni2005}.}
    \label{tab:cortical-coords-params}
\end{table}

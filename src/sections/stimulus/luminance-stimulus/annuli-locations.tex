\subsubsection{Annuli locations}
\label{sec:annuli-locations}

The distance between the centers of neighboring annuli is
\begin{equation}
    \anndist = \ceil{\anndistscale \cdot \diam},
\end{equation}
where $k \in [1, \infty)$ is the distance between annuli in number of diameters.
The variable $\anndist \in \mathbb{N}$ then also indicates the number of pixels allocated to each annulus and void around it. Consequently, we can determine the number of annuli that are (partially) located in each row $\repsinrow \in \mathbb{N}$ and column $\repsincol \in \mathbb{N}$ of the stimulus:
\begin{equation}
    \repsinrow = \ceil{\frac{\fullwidth}{\anndist}}, \;
    \repsincol = \ceil{\frac{\fullheight}{\anndist}}. 
\end{equation}

Let $\annuliset$ be the set containing the top left coordinates of all stimulus' annuli:
\begin{equation}
\begin{gathered}
    \annuliset  \coloneqq \bigcup_{i = 0}^{\repsinrow - 1} \bigcup_{i = 0}^{\repsincol - 1}
    \annstart_{i, j}, \\
    \annstart_{i, j} = \left( 
        i \cdot \anndist + \floor{ \frac{1}{2} (\anndist - \diam) }, \ 
        j \cdot \anndist + \floor{ \frac{1}{2} (\anndist - \diam) }
    \right).
\end{gathered}
\end{equation}

As we position the annuli in the middle of the allocated space, for each stimulus, there exist values of $\anndistscale$, such that the last annulus in a row or column starts outside of the bounds of the stimulus. Such annuli need to be removed from the set $\annuliset$:
\begin{equation}
    \annuliset \leftarrow \annuliset \setminus \{ (\annstart_x, \annstart_y) \in \annuliset \ | \ (\annstart_x > \fullwidth) \land (\annstart_y > \fullheight) \}.
\end{equation}
\subsubsection{Figure composition}
\label{sec:figure-composition}

Let $\fullmatrix \in [0, 1]^{\fullwidth \times \fullheight}$ be the matrix allocated for the stimulus' luminance values, where $\fullwidth, \fullheight \in [\diam, \infty)$ represent the stimulus width and height, respectively.
Let $\figwidth \in (0, \frac{1}{2} \fullwidth]$ be the \stimfig{} width, and $\figheight \in (0, \frac{1}{2} \fullheight]$ - its height. The \stimfig{} comprises a submatrix of the full stimulus $\fullmatrix$. The center of the \stimfig{} is located at the pixel distance $\figecc$ from the stimulus center, as defined in Equation (\ref{eq:figdiag}), at an angle $-\figangle$ to the horizontal axis, where $\figangle \sim U(\figanglemin, \figanglemax)$. Since the \stimfig{} is located in the bottom right quadrant,
\begin{align}
    %\begin{aligned}
        \label{eq:figangle-min}
        \figanglemin &=
        \begin{cases}
            \cos^{-1} \left(
                \frac{\fullwidth - \figwidth}{2 \cdot \figecc}
            \right)
            &\text{ if } \figecc > \sqrt{ \left( \frac{\fullwidth - \figwidth}{2} \right)^2 + \left( \frac{\figheight}{2} \right)^2 } \\
            \sin^{-1} \left( 
                \frac{\figheight}{2 \cdot \figecc}
            \right)
            &\text{ otherwise}
        \end{cases}, \\
        \label{eq:figangle-max}
        \figanglemax &= \frac{\pi}{2} - 
        \begin{cases}
            \cos^{-1} \left(
                \frac{\fullheight - \figheight}{2 \cdot \figecc}
            \right)
            &\text{ if } \figecc > \sqrt{ \left( \frac{\fullwidth - \figwidth}{2} \right)^2 + \left( \frac{\figheight}{2} \right)^2 } \\
            \sin^{-1} \left( 
                \frac{\figwidth}{2 \cdot \figecc}
            \right)
            &\text{ otherwise}
        \end{cases}.
    %\end{aligned}
\end{align}

In Equations (\ref{eq:figangle-min})-(\ref{eq:figangle-max}), the angles depend on the eccentricity $\figecc$. The visualization of the minimum angles $\figanglemin$ for various eccentricities can be seen in Figure \ref{fig:figangles-min}. 
In Figure \ref{fig:figangles-min1}, the eccentricity is too large for the figure to be adjacent to the horizontal axis (first case in Equation (\ref{eq:figangle-min})); 
in Figure \ref{fig:figangles-min2}, it is small enough to touch the horizontal axis and large enough to reach the right edge of the stimulus (edge case); 
in Figure \ref{fig:figangles-min3}, the eccentricity is too small to reach the edge (second case).
The maximum angle varies with eccentricity similar to the minimum angle.

\begin{figure}[!htp]
    \centering
    \begin{subfigure}[t]{0.3\textwidth}
        \centering
        % ----- INPUT
\newcommand{\angoneimagew}{3.7}
\newcommand{\angoneimageh}{6}
% \newcommand{\angoneimagew}{\textwidth}
% \newcommand{\angoneimageh}{0.8 * \angoneimagew}

% figure dimensions
\newcommand{\angonefigw}{0.5 * \angoneimagew}
\newcommand{\angonefigh}{0.85 * \angoneimagew}

% shifts
\newcommand{\angoneshiftw}{0.1 * \angoneimagew}
\newcommand{\angoneshifth}{0.06 * \angoneimageh}

% ----- misc
\newcommand{\angonecenterx}{0}
\newcommand{\angonecentery}{0}
% \newcommand{\angonecenterx}{0.5 * \angoneimagew}
% \newcommand{\angonecentery}{0.5 * \angoneimageh}

% figure center
\newcommand{\angonefigcenterx}{\angoneimagew - 0.5 * \angonefigw}
\newcommand{\angonefigcentery}{-\angonecentery - 1.2 * \angonefigh}

% figure coords
\newcommand{\angonefigleftx}{\angonefigcenterx - 0.5 * \angonefigw}
\newcommand{\angonefigrightx}{\angonefigcenterx + 0.5 * \angonefigw}
\newcommand{\angonefigtopy}{\angonefigcentery + 0.5 * \angonefigh}
\newcommand{\angonefigbottomy}{\angonefigcentery - 0.5 * \angonefigh}

\begin{tikzpicture}[
        arr/.style = { -{Stealth[ ]} },
        whitedoublearrow/.style = {>=stealth, draw=color-three, fill=color-three, very thick, <->},
        pinkarrow/.style = {arr, draw=color-three-light, fill=color-three-light, very thick},
        yellowframe/.style = {rounded corners=0.2cm, very thick, color-one, fill=color-one, fill opacity=0.1},
        blueframe/.style = {rounded corners=0.2cm, very thick, color-two-light, fill=color-two-light, fill opacity=0.1},
        anglearrow/.style = {draw, >=stealth, color-three, very thick, ->, angle eccentricity=2.3},
        cnode/.style = {circle,thick,fill=color-one},
    ]
    
    \begin{scope}

        % space line
        \draw[color=white] (\angonecenterx - 2 * \angoneshiftw, --\angonecentery) -- (\angoneimagew + 2 * \angoneshiftw, -\angoneimageh);

        % quadrant
        \draw[yellowframe] (\angonecenterx, -\angonecentery) rectangle (\angoneimagew, -\angoneimageh) {};
        
        % quadrant width
        \node[shift={(0, -1 * \angoneshifth)}] (startquadw) at (\angonecenterx, -\angoneimageh) {};
        \node[shift={(0, -1 * \angoneshifth)}] (endquadw) at (\angoneimagew, -\angoneimageh) {};

        \path[whitedoublearrow] (startquadw) edge node {\bgcolorsmalltext{white}{black}{$\fullwidth / 2$}} (endquadw);

        \draw[very thick, color-three] (\angonecenterx, -\angoneimageh) -- (\angonecenterx, -\angoneimageh - 2 * \angoneshifth);
        \draw[very thick, color-three] (\angoneimagew, -\angoneimageh) -- (\angoneimagew, -\angoneimageh - 2 * \angoneshifth);

        % quadrant height
        \node[shift={(-1 * \angoneshiftw, 0)}] (startquadh) at (\angonecenterx, -\angonecentery) {};
        \node[shift={(-1 * \angoneshiftw, 0)}] (endquadh) at (\angonecenterx, -\angoneimageh) {};

        \path[whitedoublearrow] (startquadh) edge node[rotate=90] {\bgcolorsmalltext{white}{black}{$\fullheight / 2$}} (endquadh);

        \draw[very thick, color-three] (\angonecenterx, -\angoneimageh) -- (\angonecenterx - 2 * \angoneshiftw, -\angoneimageh);
        \draw[very thick, color-three] (\angonecenterx - 2 * \angoneshiftw, -\angonecentery) -- (\angonecenterx, -\angonecentery);
        
        % stimulsus center
        \node[style=cnode] (cent) at (\angonecenterx, \angonecentery) {};
        
        % figure
        \draw[blueframe] (\angonefigleftx, -\angonefigtopy) rectangle (\angonefigrightx, -\angonefigbottomy) {};
        
        % figure width
        \node[shift={(0, -1 * \angoneshifth)}] (startfigw) at (\angonefigleftx, -\angonefigbottomy) {};
        \node[shift={(0, -1 * \angoneshifth)}] (endfigw) at (\angonefigrightx, -\angonefigbottomy) {};
        
        \path[whitedoublearrow] (startfigw) edge node {\bgcolorsmalltext{white}{black}{$\figwidth$}} (endfigw);
        
        \draw[very thick, color-three] (\angonefigleftx, -\angonefigbottomy) -- (\angonefigleftx, -\angonefigbottomy - 2 * \angoneshifth);
        \draw[very thick, color-three] (\angonefigrightx, -\angonefigbottomy) -- (\angonefigrightx, -\angonefigbottomy - 2 * \angoneshifth);
        
        % figure height
        \node[shift={(\angoneshiftw, 0)}] (startfigh) at (\angonefigrightx, -\angonefigtopy) {};
        \node[shift={(\angoneshiftw, 0)}] (endfigh) at (\angonefigrightx, -\angonefigbottomy) {};
        
        \path[whitedoublearrow] (startfigh) edge node[rotate=90] {\bgcolorsmalltext{white}{black}{$\figheight$}} (endfigh);
        
        \draw[very thick, color-three] (\angonefigrightx, -\angonefigtopy) -- (\angonefigrightx + 2 * \angoneshiftw, -\angonefigtopy);
        \draw[very thick, color-three] (\angonefigrightx, -\angonefigbottomy) -- (\angonefigrightx + 2 * \angoneshiftw, -\angonefigbottomy);
        
        % center of figure
        \node[style=cnode] (cent2) at (\angonefigcenterx, -\angonefigcentery) {};
        
        % eccentricity
        \path[whitedoublearrow] (cent) edge node[sloped] {\bgcolorsmalltext{white}{black}{$\figecc$}} (cent2);
        
        % angle
        \pic[anglearrow, "\bgcolorsmalltext{white}{black}{$\; \figanglemin \;$}"] {angle = cent2--cent--xright};
        
        
    \end{scope}
\end{tikzpicture}
        \vspace{-1em}
        \caption{$\figecc > \sqrt{ \left( \frac{\fullwidth - \figwidth}{2} \right)^2 + \left( \frac{\figheight}{2} \right)^2 }$.}
        \label{fig:figangles-min1}
    \end{subfigure}
    \hspace{0.03\textwidth}
    \begin{subfigure}[t]{0.3\textwidth}
        \centering
        % ----- INPUT
\newcommand{\angtwoimagew}{3.7}
\newcommand{\angtwoimageh}{6}
% \newcommand{\angtwoimagew}{\textwidth}
% \newcommand{\angtwoimageh}{0.8 * \angtwoimagew}

% figure dimensions
\newcommand{\angtwofigw}{0.5 * \angtwoimagew}
\newcommand{\angtwofigh}{0.85 * \angtwoimagew}

% shifts
\newcommand{\angtwoshiftw}{0.1 * \angtwoimagew}
\newcommand{\angtwoshifth}{0.06 * \angtwoimageh}

% ----- misc
\newcommand{\angtwocenterx}{0}
\newcommand{\angtwocentery}{0}
% \newcommand{\angtwocenterx}{0.5 * \angtwoimagew}
% \newcommand{\angtwocentery}{0.5 * \angtwoimageh}

% figure center
\newcommand{\angtwofigcenterx}{\angtwoimagew - 0.5 * \angtwofigw}
\newcommand{\angtwofigcentery}{-\angtwocentery - 0.5 * \angtwofigh}

% figure coords
\newcommand{\angtwofigleftx}{\angtwofigcenterx - 0.5 * \angtwofigw}
\newcommand{\angtwofigrightx}{\angtwofigcenterx + 0.5 * \angtwofigw}
\newcommand{\angtwofigtopy}{\angtwofigcentery + 0.5 * \angtwofigh}
\newcommand{\angtwofigbottomy}{\angtwofigcentery - 0.5 * \angtwofigh}

\begin{tikzpicture}[
        arr/.style = { -{Stealth[ ]} },
        whitedoublearrow/.style = {>=stealth, draw=color-three, fill=color-three, very thick, <->},
        pinkarrow/.style = {arr, draw=color-three-light, fill=color-three-light, very thick},
        yellowframe/.style = {rounded corners=0.2cm, very thick, color-one, fill=color-one, fill opacity=0.1},
        blueframe/.style = {rounded corners=0.2cm, very thick, color-two-light, fill=color-two-light, fill opacity=0.1},
        anglearrow/.style = {draw, >=stealth, color-three, very thick, ->, angle eccentricity=2.3},
        cnode/.style = {circle,thick,fill=color-one},
    ]
    
    \begin{scope}

        % space line
        \draw[color=white] (\angtwocenterx - 2 * \angtwoshiftw, --\angtwocentery) -- (\angtwoimagew + 2 * \angtwoshiftw, -\angtwoimageh);

        % quadrant
        \draw[yellowframe] (\angtwocenterx, -\angtwocentery) rectangle (\angtwoimagew, -\angtwoimageh) {};
        
        % quadrant width
        \node[shift={(0, -1 * \angtwoshifth)}] (startquadw) at (\angtwocenterx, -\angtwoimageh) {};
        \node[shift={(0, -1 * \angtwoshifth)}] (endquadw) at (\angtwoimagew, -\angtwoimageh) {};

        \path[whitedoublearrow] (startquadw) edge node {\bgcolorsmalltext{white}{black}{$\fullwidth / 2$}} (endquadw);

        \draw[very thick, color-three] (\angtwocenterx, -\angtwoimageh) -- (\angtwocenterx, -\angtwoimageh - 2 * \angtwoshifth);
        \draw[very thick, color-three] (\angtwoimagew, -\angtwoimageh) -- (\angtwoimagew, -\angtwoimageh - 2 * \angtwoshifth);

        % quadrant height
        \node[shift={(-1 * \angtwoshiftw, 0)}] (startquadh) at (\angtwocenterx, -\angtwocentery) {};
        \node[shift={(-1 * \angtwoshiftw, 0)}] (endquadh) at (\angtwocenterx, -\angtwoimageh) {};

        \path[whitedoublearrow] (startquadh) edge node[rotate=90] {\bgcolorsmalltext{white}{black}{$\fullheight / 2$}} (endquadh);

        \draw[very thick, color-three] (\angtwocenterx, -\angtwoimageh) -- (\angtwocenterx - 2 * \angtwoshiftw, -\angtwoimageh);
        \draw[very thick, color-three] (\angtwocenterx - 2 * \angtwoshiftw, -\angtwocentery) -- (\angtwocenterx, -\angtwocentery);
        
        % stimulsus center
        \node[style=cnode] (cent) at (\angtwocenterx, \angtwocentery) {};
        
        % figure
        \draw[blueframe] (\angtwofigleftx, -\angtwofigtopy) rectangle (\angtwofigrightx, -\angtwofigbottomy) {};
        
        % figure width
        \node[shift={(0, -1 * \angtwoshifth)}] (startfigw) at (\angtwofigleftx, -\angtwofigbottomy) {};
        \node[shift={(0, -1 * \angtwoshifth)}] (endfigw) at (\angtwofigrightx, -\angtwofigbottomy) {};
        
        \path[whitedoublearrow] (startfigw) edge node {\bgcolorsmalltext{white}{black}{$\figwidth$}} (endfigw);
        
        \draw[very thick, color-three] (\angtwofigleftx, -\angtwofigbottomy) -- (\angtwofigleftx, -\angtwofigbottomy - 2 * \angtwoshifth);
        \draw[very thick, color-three] (\angtwofigrightx, -\angtwofigbottomy) -- (\angtwofigrightx, -\angtwofigbottomy - 2 * \angtwoshifth);
        
        % figure height
        \node[shift={(\angtwoshiftw, 0)}] (startfigh) at (\angtwofigrightx, -\angtwofigtopy) {};
        \node[shift={(\angtwoshiftw, 0)}] (endfigh) at (\angtwofigrightx, -\angtwofigbottomy) {};
        
        \path[whitedoublearrow] (startfigh) edge node[rotate=90] {\bgcolorsmalltext{white}{black}{$\figheight$}} (endfigh);
        
        \draw[very thick, color-three] (\angtwofigrightx, -\angtwofigtopy) -- (\angtwofigrightx + 2 * \angtwoshiftw, -\angtwofigtopy);
        \draw[very thick, color-three] (\angtwofigrightx, -\angtwofigbottomy) -- (\angtwofigrightx + 2 * \angtwoshiftw, -\angtwofigbottomy);
        
        % center of figure
        \node[style=cnode] (cent2) at (\angtwofigcenterx, -\angtwofigcentery) {};
        
        % eccentricity
        \path[whitedoublearrow] (cent) edge node[sloped] {\bgcolorsmalltext{white}{black}{$\figecc$}} (cent2);
        
        % angle
        \pic[anglearrow, "\bgcolorsmalltext{white}{black}{$\; \figanglemin \;$}"] {angle = cent2--cent--xright};
        
        
    \end{scope}
\end{tikzpicture}
        \vspace{-1em}
        \caption{$\figecc = \sqrt{ \left( \frac{\fullwidth - \figwidth}{2} \right)^2 + \left( \frac{\figheight}{2} \right)^2 }$.}
        \label{fig:figangles-min2}
    \end{subfigure}
    \hspace{0.03\textwidth}
    \begin{subfigure}[t]{0.3\textwidth}
        \centering
        % ----- INPUT
\newcommand{\angthreeimagew}{3.7}
\newcommand{\angthreeimageh}{6}
% \newcommand{\angthreeimagew}{\textwidth}
% \newcommand{\angthreeimageh}{0.8 * \angthreeimagew}

% figure dimensions
\newcommand{\angthreefigw}{0.5 * \angthreeimagew}
\newcommand{\angthreefigh}{0.85 * \angthreeimagew}

% shifts
\newcommand{\angthreeshiftw}{0.1 * \angthreeimagew}
\newcommand{\angthreeshifth}{0.06 * \angthreeimageh}

% ----- misc
\newcommand{\angthreecenterx}{0}
\newcommand{\angthreecentery}{0}
% \newcommand{\angthreecenterx}{0.5 * \angthreeimagew}
% \newcommand{\angthreecentery}{0.5 * \angthreeimageh}

% figure center
\newcommand{\angthreefigcenterx}{\angthreeimagew - 1 * \angthreefigw}
\newcommand{\angthreefigcentery}{-\angthreecentery - 0.5 * \angthreefigh}

% figure coords
\newcommand{\angthreefigleftx}{\angthreefigcenterx - 0.5 * \angthreefigw}
\newcommand{\angthreefigrightx}{\angthreefigcenterx + 0.5 * \angthreefigw}
\newcommand{\angthreefigtopy}{\angthreefigcentery + 0.5 * \angthreefigh}
\newcommand{\angthreefigbottomy}{\angthreefigcentery - 0.5 * \angthreefigh}

\begin{tikzpicture}[
        arr/.style = { -{Stealth[ ]} },
        whitedoublearrow/.style = {>=stealth, draw=color-three, fill=color-three, very thick, <->},
        pinkarrow/.style = {arr, draw=color-three-light, fill=color-three-light, very thick},
        yellowframe/.style = {rounded corners=0.2cm, very thick, color-one, fill=color-one, fill opacity=0.1},
        blueframe/.style = {rounded corners=0.2cm, very thick, color-two-light, fill=color-two-light, fill opacity=0.1},
        anglearrow/.style = {draw, >=stealth, color-three, very thick, ->, angle eccentricity=2.3},
        cnode/.style = {circle,thick,fill=color-one},
    ]
    
    \begin{scope}

        % space line
        \draw[color=white] (\angthreecenterx - 2 * \angthreeshiftw, --\angthreecentery) -- (\angthreeimagew + 2 * \angthreeshiftw, -\angthreeimageh);

        % quadrant
        \draw[yellowframe] (\angthreecenterx, -\angthreecentery) rectangle (\angthreeimagew, -\angthreeimageh) {};
        
        % quadrant width
        \node[shift={(0, -1 * \angthreeshifth)}] (startquadw) at (\angthreecenterx, -\angthreeimageh) {};
        \node[shift={(0, -1 * \angthreeshifth)}] (endquadw) at (\angthreeimagew, -\angthreeimageh) {};

        \path[whitedoublearrow] (startquadw) edge node {\bgcolorsmalltext{white}{black}{$\fullwidth / 2$}} (endquadw);

        \draw[very thick, color-three] (\angthreecenterx, -\angthreeimageh) -- (\angthreecenterx, -\angthreeimageh - 2 * \angthreeshifth);
        \draw[very thick, color-three] (\angthreeimagew, -\angthreeimageh) -- (\angthreeimagew, -\angthreeimageh - 2 * \angthreeshifth);

        % quadrant height
        \node[shift={(-1 * \angthreeshiftw, 0)}] (startquadh) at (\angthreecenterx, -\angthreecentery) {};
        \node[shift={(-1 * \angthreeshiftw, 0)}] (endquadh) at (\angthreecenterx, -\angthreeimageh) {};

        \path[whitedoublearrow] (startquadh) edge node[rotate=90] {\bgcolorsmalltext{white}{black}{$\fullheight / 2$}} (endquadh);

        \draw[very thick, color-three] (\angthreecenterx, -\angthreeimageh) -- (\angthreecenterx - 2 * \angthreeshiftw, -\angthreeimageh);
        \draw[very thick, color-three] (\angthreecenterx - 2 * \angthreeshiftw, -\angthreecentery) -- (\angthreecenterx, -\angthreecentery);
        
        % stimulsus center
        \node[style=cnode] (cent) at (\angthreecenterx, \angthreecentery) {};
        
        % figure
        \draw[blueframe] (\angthreefigleftx, -\angthreefigtopy) rectangle (\angthreefigrightx, -\angthreefigbottomy) {};
        
        % figure width
        \node[shift={(0, -1 * \angthreeshifth)}] (startfigw) at (\angthreefigleftx, -\angthreefigbottomy) {};
        \node[shift={(0, -1 * \angthreeshifth)}] (endfigw) at (\angthreefigrightx, -\angthreefigbottomy) {};
        
        \path[whitedoublearrow] (startfigw) edge node {\bgcolorsmalltext{white}{black}{$\figwidth$}} (endfigw);
        
        \draw[very thick, color-three] (\angthreefigleftx, -\angthreefigbottomy) -- (\angthreefigleftx, -\angthreefigbottomy - 2 * \angthreeshifth);
        \draw[very thick, color-three] (\angthreefigrightx, -\angthreefigbottomy) -- (\angthreefigrightx, -\angthreefigbottomy - 2 * \angthreeshifth);
        
        % figure height
        \node[shift={(\angthreeshiftw, 0)}] (startfigh) at (\angthreefigrightx, -\angthreefigtopy) {};
        \node[shift={(\angthreeshiftw, 0)}] (endfigh) at (\angthreefigrightx, -\angthreefigbottomy) {};
        
        \path[whitedoublearrow] (startfigh) edge node[rotate=90] {\bgcolorsmalltext{white}{black}{$\figheight$}} (endfigh);
        
        \draw[very thick, color-three] (\angthreefigrightx, -\angthreefigtopy) -- (\angthreefigrightx + 2 * \angthreeshiftw, -\angthreefigtopy);
        \draw[very thick, color-three] (\angthreefigrightx, -\angthreefigbottomy) -- (\angthreefigrightx + 2 * \angthreeshiftw, -\angthreefigbottomy);
        
        % center of figure
        \node[style=cnode] (cent2) at (\angthreefigcenterx, -\angthreefigcentery) {};
        
        % eccentricity
        \path[whitedoublearrow] (cent) edge node[sloped] {\bgcolorsmalltext{white}{black}{$\figecc$}} (cent2);
        
        % angle
        \pic[anglearrow, "\bgcolorsmalltext{white}{black}{$\; \figanglemin \;$}"] {angle = cent2--cent--xright};
        
        
    \end{scope}
\end{tikzpicture}
        \vspace{-1em}
        \caption{$\figecc < \sqrt{ \left( \frac{\fullwidth - \figwidth}{2} \right)^2 + \left( \frac{\figheight}{2} \right)^2 }$.}
        \label{fig:figangles-min3}
    \end{subfigure}
    \caption[Figure angle and eccentricity]{Minimum angle $\figanglemin$ between the horizontal axis and the center of the \stimfig{} for different values of eccentricity $\figecc$. The red area represents the lower bottom quadrant of the stimulus, the green area represents the \stimfig.}
    \label{fig:figangles-min}
\end{figure}

Now, we can locate the center of the \stimfig{} $\figcenter \in \mathbb{R}_+^2$:
\begin{equation}
    \figcenter = \left(
        \frac{1}{2} \fullwidth + \figecc \cdot \cos(\figangle), \
        \frac{1}{2} \fullheight + \figecc \cdot \sin(\figangle)
    \right).
\end{equation}
The center of the \stimfig{} need not be integer as it is only used for determining the \stimfig{}'s top left ($\figstart \in \mathbb{N}^2$) and bottom right ($\figfinish \in \mathbb{N}^2$) coordinates:
\begin{align}
\begin{split}
    \figstart &= \ceil{\figcenter - \frac{1}{2} (\figheight, \figwidth)}, \\
    \figfinish &= \figstart + (\figheight, \figwidth),
\end{split}
\label{eq:figure-location}
\end{align}
The ceiling function in Equation (\ref{eq:figure-location}) is applied element-wise.
\subsubsection{Figure composition}
\label{sec:figure-composition}

Let $\fullmatrix \in [0, 1]^{\fullwidth \times \fullheight}$ be the matrix allocated for the stimulus' luminance values, where $\fullwidth, \fullheight \in [\diam, \infty)$ represent the stimulus width and height, respectively.
Let $\figwidth \in (0, \frac{1}{2} \fullwidthdg]$ be the \stimfig{} width, and $\figheight \in (0, \frac{1}{2} \fullheightdg]$ - its height. The \stimfig{} comprises a submatrix of the full stimulus $\fullmatrix$. The \stimfig{} is located at the pixel distance $\figecc$ from the stimulus center, as defined in Equation (\ref{eq:figdiag}), at an angle $\figangle \sim U(\figanglemin, \figanglemax)$ to the horizontal axis. Since the \stimfig{} is located in the bottom right quadrant,
\begin{equation}
    \begin{aligned}
        \figanglemin &=
        \begin{cases}
            \cos^{-1} \left(
                \frac{\fullwidth - \figwidth}{2 \cdot \figecc}
            \right)
            &\text{ if } \figecc > \frac{\fullwidth - \figwidth}{2} \\
            \sin^{-1} \left( 
                \frac{\figheight}{2 \cdot \figecc}
            \right)
            &\text{ otherwise}
        \end{cases}, \\
        \figanglemax &= \frac{\pi}{2} - 
        \begin{cases}
            \cos^{-1} \left(
                \frac{\fullheight - \figheight}{2 \cdot \figecc}
            \right)
            &\text{ if } \figecc > \frac{\fullheight - \figheight}{2} \\
            \sin^{-1} \left( 
                \frac{\figwidth}{2 \cdot \figecc}
            \right)
            &\text{ otherwise}
        \end{cases}.
    \end{aligned}
\end{equation}
Now, we can locate the center of the \stimfig{} $\figcenter \in \mathbb{R}_+^2$:
\begin{equation}
    \figcenter = \left(
        \frac{1}{2} \fullwidth + \figecc \cdot \cos(\figangle), \
        \frac{1}{2} \fullheight + \figecc \cdot \sin(\figangle)
    \right).
\end{equation}
The center of the \stimfig{} need not be integer as it is only used for determining the \stimfig{}'s top left ($\figstart \in \mathbb{N}^2$) and bottom right ($\figfinish \in \mathbb{N}^2$) coordinates:
\begin{align}
\begin{split}
    \figstart &= \ceil{\figcenter - \frac{1}{2} (\figheight, \figwidth)}, \\
    \figfinish &= \figstart + (\figheight, \figwidth),
\end{split}
\label{eq:figure-location}
\end{align}
The ceiling function in Equation (\ref{eq:figure-location}) is applied element-wise.
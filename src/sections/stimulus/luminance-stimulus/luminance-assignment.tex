\subsubsection{Luminance assignment}
\label{sec:luminance-assignment}

First, all values of the stimulus $\fullmatrix$ are initialized with the void luminance value $\vlum$. Then, the annuli with relevant contrasts are added to the matrix. Let $\view: \annuliset \to \{ \fig, \ground \}$ be the function mapping an annulus to its view. Let $\annstart \in \annuliset$, then
\begin{equation}
    \view(\annstart) = 
    \begin{cases}
        \fig & \text{ if } 
        \{ 
        \annstart, 
        \cdots,
        \annstart + (\diam, \diam)
        \}
        \cap
        \{ \figstart, \cdots, \figfinish \}
        \neq \emptyset \\
        \ground & \text{ otherwise} 
    \end{cases}.
\end{equation}
So, an annulus is said to be located in the \stimfig{} if it is (partially) contained in it, i.e., the intersection of pixels belonging to the \stimfig{} and the annulus is non-empty.

Let $\unirand: \annuliset \to U(0, 1)$ be a random value that varies per annulus. The values are independent and identically distributed (i.i.d.). Then, for all $\annstart \in \annuliset$ and $i, j \in \{ 0, 1, \cdots, \diam - 1\}$,
\begin{equation}
    \fullmatrix_{\ \annstart + (i, j)} \leftarrow \vlum +
    \begin{cases}
        (\annmatrix_{i, j} - \vlum) \cdot \left( \unirand(\annstart) \cdot \contrange + \vlum \cdot (1 - \contrange) \right) 
        &\text{ if } \view(\annstart) = \fig  \\
         (\annmatrix_{i, j} - \vlum) \cdot \unirand(\annstart) 
        &\text{ otherwise}
    \end{cases},
\end{equation}
where $\contrange \in (0, 1]$ is the range of annuli contrast in the \stimfig.
Now, the stimulus luminance matrix is complete.
Thus, the luminance of the void, similar to annuli's edges as demonstrated in Equation (\ref{eq:grating}), are still assigned the luminance of $\vlum$; the mean of all pixels' luminance values is also equal to $\vlum$. An example of the resulting luminance matrix' binary heat map is visualized in Figure \ref{fig:full-stimulus-example}.
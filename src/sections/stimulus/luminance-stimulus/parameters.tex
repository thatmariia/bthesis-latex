\subsubsection{Input parameters}
\label{sec:stimulus-parameters}

The appearance of the stimulus depends on many input parameters.
The reader may omit this section as the parameters are explained as they appear in the text. The summary list below is presented for convenience.

All spatial parameters are given in visual degrees: $1^\circ$ is equivalent to $1$ cm when the distance between the visual field and an eye is $37.5$ cm. For one such parameter (annulus diameter), the input is provided in pixels, too, allowing for scaling other parameters. The values given in visual degrees are further referred to as physical, and the pixel values - naturally, as pixel.

For the construction of an annulus, the following parameters are given:
\begin{itemize}
    \item $\spatfreq \in \mathbb{R}_+$ - the spatial frequency in cycles per degree;
    
    \item $\vlum \in [0, 1]$ - the void luminance value;
    
    \item $\diamdg \in \mathbb{R}_+$ - the diameter of a single annulus in visual degrees;
    
    \item $\diam \in \mathbb{N}$ - the diameter of a single annulus in pixels.
\end{itemize}
Since the mapping of the physical space (degrees) to pixel values in isomorphic, a conversion coefficient $\atopix \in \mathbb{R}_+$ between the two can be defined as follows:
\begin{equation}
    \atopix = \frac{\diam}{\diamdg}.
\end{equation}

The parameters relevant for the full stimulus are
\begin{itemize}
    \item $\anndistscale \in [1, \infty)$ - the scalar that represents the distance scale between neighboring annuli (the minimum possible value is $1$ to ensure the annuli do not overlap);
    
    \item $\fullwidthdg \in [\diam, \infty)$ - the width of the stimulus in visual degrees; in pixels, the width is
    $
        \fullwidth = \ceil{\atopix \cdot \fullwidthdg};
    $
    
    \item $\fullheightdg \in [\diam, \infty)$ - the height of the stimulus in visual degrees; in pixels, the height is
    $
        \fullheight = \ceil{\atopix \cdot \fullheightdg};
    $
\end{itemize}

The parameters relevant for the figure are
\begin{itemize}

    \item $\contrange \in (0, 1]$ - the range of annuli contrasts in the figure;
    
    \item $\figwidthdg \in (0, \frac{1}{2} \fullwidthdg]$ - the width of the figure in visual degrees (must fit in the bottom right quadrant of the stimulus); in pixels, the width is
    $
        \figwidth = \ceil{\atopix \cdot \figwidthdg};
    $
    
    \item $\figheightdg \in (0, \frac{1}{2} \fullheightdg]$ - the height of the figure in visual degrees; in pixels, the width is
    $
        \figheight = \ceil{\atopix \cdot \figheightdg};
    $
    
    \item $\figeccdg \in [\figdiag, \| (\frac{1}{2} \fullwidthdg, \frac{1}{2} \fullheightdg)\| - \figdiag]$, where
    \begin{equation}
        \figdiag = \left\| 
            \left(
                \frac{1}{2} \figwidthdg, 
                \frac{1}{2} \figheightdg 
            \right)
        \right\|,
        \label{eq:figdiag}
    \end{equation}
    - the distance between the center of the stimulus and the center of the figure in visual degrees; in pixels, this distance is
    $
        \figecc = \ceil{\atopix \cdot \figeccdg}.
    $
\end{itemize}

Finally, the parameter for the stimulus patch is
\begin{itemize}
    \item $\patchsizedg \in (0, \min(\figwidthdg, \figheightdg)]$ - the side length of the stimulus patch in visual degrees; in pixels, the length is
    $
        \patchsize = \ceil{\atopix \cdot \patchsizedg}.
    $
    Additionally, for further processing, $(n | \patchsize)$ must hold.
\end{itemize}

The parameters' values used for composing the stimuli are displayed in Table \ref{tab:stimulus-composition-params}.

\begin{table}[!htp]
    \centering
    \begin{tabular}{|
>{\columncolor{main-color}}c |ccccc|}
\hline
\textbf{Parameter}      & \multicolumn{5}{c|}{\cellcolor{main-color}\textbf{Value(s)}}                                                                      \\ \hline
\textbf{$\pmb{\hat{d}}$}      & \multicolumn{5}{c|}{$0.7^\circ$}                                                                                                    \\ \hline
$\pmb{\vlum}$           & \multicolumn{5}{c|}{$0.5$}                                                                                                          \\ \hline
\textbf{$\pmb{\Omega}$} & \multicolumn{5}{c|}{$5.7$ cycles/degree}                                                                                            \\ \hline
$\pmb{\hat{\ell_{F_w}}}$      & \multicolumn{5}{c|}{$(9 \pm 0.7)^\circ$}                                                                                            \\ \hline
$\pmb{\hat{\ell_{F_h}}}$      & \multicolumn{5}{c|}{$(5 \pm 0.4)^\circ$}                                                                                            \\ \hline
$\pmb{\hat{\ecc_F}}$          & \multicolumn{5}{c|}{$(7 \pm 1)^\circ$}                                                                                              \\ \hline
$\pmb{k}$               & \multicolumn{1}{c|}{$1$}    & \multicolumn{1}{c|}{$1.125$}  & \multicolumn{1}{c|}{$1.250$}  & \multicolumn{1}{c|}{$1.375$}  & $1.5$ \\ \hline
$\pmb{\eta}$            & \multicolumn{1}{c|}{$0.01$} & \multicolumn{1}{c|}{$0.0257$} & \multicolumn{1}{c|}{$0.5050$} & \multicolumn{1}{c|}{$0.7525$} & $1$   \\ \hline
\end{tabular}
    \caption[Stimuli parameters]{The values of the parameters used to generate texture stimuli. The parameters are chosen to be either identical or close to \cite{MaryamPLACEHOLDER}.}
    \label{tab:stimulus-composition-params}
\end{table}
\subsubsection{Patch selection}
\label{sec:patch-selection}

A square patch $\patchmatrix \in [0, 1]^{\patchsize \times \patchsize}$ concentric to the \stimfig{} with the side length $\patchsize \in (0, \min(\fullwidth, \fullheight)]$,  such that $n | \patchsize$, is selected. It can be defined as
\begin{equation}
    \patchmatrix =
    \begin{pmatrix}
        \fullmatrix_{\ \patchstart} & \cdots  & \fullmatrix_{\ \patchstart + (\patchsize, 0)}  \\
        \vdots & \ddots & \vdots  \\
        \fullmatrix_{\ \patchstart + (0, \patchsize)} & \cdots  &  \fullmatrix_{\ \patchstart + (\patchsize, \patchsize)}
    \end{pmatrix}, \;\
    \patchstart = \left( \ceil{ \figcenter - \frac{\patchsize}{2}}, \ceil{  \figcenter - \frac{\patchsize}{2} } \right).
\end{equation}
An example of the resulting binary heat map of the texture stimulus patch is visualized in Figure \ref{fig:stim-patch-example}.

\begin{figure}
    \centering
    \includegraphics[width=0.4\textwidth]{src/assets/images/stimulus-patch.png}
    \caption[Stimulus patch]{The stimulus patch obtained from Figure \ref{fig:full-stimulus-example}.}
    \label{fig:stim-patch-example}
\end{figure}






